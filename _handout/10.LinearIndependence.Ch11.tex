% \documentclass[a4paper,11pt]{article}
% \usepackage[hyperref]{beamerarticle}

\documentclass[final]{beamer}

\usepackage{kotex}
\usepackage{amsfonts,amsmath,xob-amssymb}

\usepackage{amsthm}
\newtheorem{defn}{Definition}
\newtheorem{thm}{Theorem}

\usepackage{cancel}
\usepackage{enumerate}

\mode<presentation>{
	\usetheme{Madrid}
	\usecolortheme{default}
	\usefonttheme{professionalfonts}
}

\def\b{\boldsymbol}

\mode<article>{
\usepackage{fullpage}
}
\usepackage{ulem}

\author[조남운]{\url{mailto:eyeofyou@korea.ac.kr}}
\title{Linear Independence}
\subtitle{Ch.11}

\begin{document}
	
\maketitle

\mode<presentation>{
\begin{frame}[t]{Table of Contents}
	\tableofcontents
\end{frame}
%--- Next Frame ---%
}

\section{Linear Independence} % (fold)
\label{sec:linear_independence}
\begin{frame}[t]{LI: Definition}
	\begin{defn}
		[Linear Combinations, Span, Linear (in)dependence]
		$\mathcal{L}$ is \uline{spanned} set generated by \uline{linear combination} of $k$ vectors $\mathbf{v_1,\cdots,v_k}$
		\[
			\mathcal{L}[\mathbf{v_1,v_2,\cdots,v_k\in\mathbb{R}^n}]:=\left\{\sum_i^k r_i \mathbf{v}_i: \forall r_i\in \mathbb{R} \right\}
		\]
		$\mathbf{v_1,\cdots,v_k}$ are \uline{linearly independent} iff:
		\[
			\mathcal{L}[\mathbf{v_1,v_2,\cdots,v_k\in\mathbb{R}^n}]\subset \mathbb{R}^k
		\]
		Otherwise, $\mathbf{v_1,\cdots,v_k}$ are \uline{linearly dependent}
	\end{defn}
	Note: Different vectors can span the same space. Canonical basis $\mathbf{e_i}$ can be the representative vector (basis) for $\mathbb{R}^n$ space.
\end{frame}
%--- Next Frame ---%

\begin{frame}[t]{LI: Alternative Definition}
	\begin{defn}
		[Linear (in)dependence: Alternative definition]
		$\mathbf{v_1,\cdots,v_k\in\mathbb{R}^n}$ are \uline{linearly dependent} iff:
		\[
			\exists c_1,\cdots,c_k \neq 0 \quad s.t.\quad \sum_i^k c_i\mathbf{v_i} = \mathbf{0}
		\]
		$\mathbf{v_1,\cdots,v_k\in\mathbb{R}^n}$ are \uline{linearly independent} iff:
		\[
			\sum_i^k c_i\mathbf{v_i} = \mathbf{0} \quad\Rightarrow\quad c_1 = \cdots = c_k = 0
		\]
	\end{defn}
\end{frame}
%--- Next Frame ---%



\begin{frame}[t]{Theorems}
	\begin{thm}
		[11.1] $\mathbf{v_1,v_2,\cdots,v_k\in\mathbb{R}^n}$ are linearly dependent iff\[
			A \begin{pmatrix}
				c_1 \\
				\vdots\\
				c_k
			\end{pmatrix} = \mathbf{0}
		\] has nonzero solution $\mathbf{c}$, where $A$ is $n\times k$ matrix whose $C_i = \mathbf{v_i}$. $i.e.,$\[
			A = \begin{pmatrix}
				\mathbf{v_1} & \mathbf{v_2} & \cdots & \mathbf{v_k}
			\end{pmatrix}
		\]
	\end{thm}
	\begin{thm}
		[11.2] $\mathbf{v_1,v_2,\cdots,v_n\in\mathbb{R}^n}$ are linearly independent iff \[
			\det\begin{pmatrix}
				\mathbf{v_1} & \mathbf{v_2} & \cdots & \mathbf{v_n}
			\end{pmatrix} \neq 0
		\]
	\end{thm}

\end{frame}
%--- Next Frame ---%

\begin{frame}[t]{Checking LI}
	\begin{block}
		{Procedure: Checking Linear (in)dependence}
		\begin{enumerate}
			\item Stack $\mathbf{v_i}$ to make $k\times n$ matrix, $A$. 
			\item Calculate $rank(A)$: this is the dimension of $\mathcal{L}[\mathbf{v_1,v_2,\cdots,v_k\in\mathbb{R}^n}]$
			\item $\mathbf{v_1,v_2,\cdots,v_k\in\mathbb{R}^n}$ are linearly independent iff $rank(A)=k$. Otherwise, they are linearly dependent
		\end{enumerate}
	\end{block}
	Note: $rank(A^T) = rank(A)$
	\begin{thm}
		[11.3] If $k>n$, any set of $k$ vectors in $\mathbb{R}^n$ is linearly dependent.
	\end{thm}
\end{frame}
%--- Next Frame ---%
% section linear_independence (end)

\section{Spanning Sets} % (fold)
\label{sec:spanning_sets}

% section spanning_sets (end)

\section{Basis and Dimension in $\mathbb{R}^n$} % (fold)
\label{sec:basis_and_dimension_in_mathbb_r_n}
\begin{frame}[t]{Basis}
	\begin{defn}
		[Basis] Let $V=\mathcal{L}[\mathbf{v_1,v_2,\cdots,v_k\in\mathbb{R}^n}]$. If $\mathbf{v_1,v_2,\cdots,v_k\in\mathbb{R}^n}$ are linearly independent, $\mathbf{v_1,v_2,\cdots,v_k\in\mathbb{R}^n}$ is called a \uline{basis} of $V$. More generally,  $\mathbf{v_1,v_2,\cdots,v_k\in\mathbb{R}^n}$ forms a \uline{basis} of $V$ if:
		\begin{enumerate}
			\item $\mathbf{v_1,v_2,\cdots,v_k\in\mathbb{R}^n}$ span $V$
			\item $\mathbf{v_1,v_2,\cdots,v_k\in\mathbb{R}^n}$ are linearly independent
		\end{enumerate}
	\end{defn}
	\begin{thm}
		[11.7] Every basis of $\mathbb{R}^n$ contains $n$ vectors.
	\end{thm}
\end{frame}
%--- Next Frame ---%

\begin{frame}[t]{Linear Independence and Basis}
	\begin{thm}
		[11.8] Let $\mathbf{v_1,v_2,\cdots,v_n\in\mathbb{R}^n}$ be a collection of $n$ vectors in $\mathbb{R}^n$. And let $n\times n$ matrix $A = \begin{pmatrix}
			\mathbf{v_1}&\cdots&\mathbf{v_n}
		\end{pmatrix}$. Then the following statements are equivalent:
		\begin{enumerate}
			\item $\mathbf{v_1,v_2,\cdots,v_n\in\mathbb{R}^n}$ are linearly independent.
			\item $\mathbf{v_1,v_2,\cdots,v_n\in\mathbb{R}^n}$ span $\mathbb{R}^n$
			\item $\mathbf{v_1,v_2,\cdots,v_n\in\mathbb{R}^n}$ form a basis of $\mathbb{R}^n$
			\item $\det(A)\neq 0$
		\end{enumerate}
	\end{thm}
\end{frame}
%--- Next Frame ---%
% section basis_and_dimension_in_mathbb_r_n (end)

\end{document}

