\documentclass[a4paper,11pt]{article}
\usepackage[hyperref]{beamerarticle}

%\documentclass[final]{beamer}

\usepackage[hangul]{kotex}
\usepackage{amsfonts,amsmath,xob-amssymb}

\usepackage{amsthm}
\newtheorem{defn}{Definition}
\newtheorem{thm}{Theorem}

\usepackage{cancel}
\usepackage{enumerate}

\mode<presentation>{
	\usetheme{Madrid}
	\usecolortheme{default}
	\usefonttheme{professionalfonts}
}

\def\b{\boldsymbol}

\mode<article>{
\usepackage{fullpage}
}
\usepackage{ulem}

\author[조남운]{\url{econMath.namun@gmail.com}}
\title{Determinants: An Overview}
\subtitle{Ch.9}

\begin{document}

\maketitle

\mode<presentation>{
\begin{frame}[t]{Table of Contents}
	\tableofcontents
\end{frame}
%--- Next Frame ---%
}

\section{The Determinant of A Matrix} % (fold)
\label{sec:the_determinant_of_a_matrix}
\subsection{Defining the Determinant} % (fold)
\label{sub:defining_the_determinant}
\begin{frame}[t]{Defining the Determinant}
	We need to define a function $\det(A): M_n \rightarrow \mathbb{R}$ satisfying
	\[
		\det A = \det(A) = \vert A\vert \begin{cases}
			= 0 \text{ if $A$ is singular}\\
			\neq 0 \text{ if $A$ is nonsingular}
		\end{cases}
	\]
	This chapter is an easy overview for determinant. For full version, see Ch.26.
	\begin{block}
		{Big Picture}
		Definition of $\det(A)$ when $A$ is
		\begin{enumerate}
			\item $1\times 1$ matrices,
			\item $2\times 2$ matrices, and
			\item $n\times n$ matrices
		\end{enumerate}
	\end{block}
\end{frame}
%--- Next Frame ---%
\begin{frame}[t]{$\det(A)$ when $A\in M_1$ and $A\in M_2$}
	\begin{defn}
		[$\det(A)$ when $A\in M_1$]
		\[
			\det(A):=A
		\]
	\end{defn}
	When $A\in M_1$, $A$ is just a scalar. $i.e.,$ $M_1 = \mathbf{R}$
	\begin{defn}
		[$\det(A)$ when $A\in M_2$]
		\[
			\det(A):= a_{11} \det(a_{22}) - a_{12}\det(a_{21}) =a_{11}a_{22}-a_{12}a_{21}
		\]
	\end{defn}
	For $M_n$, above definition is extended recursively
\end{frame}
%--- Next Frame ---%
\begin{frame}[t]{Minor and Cofactor of Matrices}
	Let $A\in M_n$
	\begin{defn}
		[Minor, Cofactor]
		Let $A_{ij}\in M_{n-1}$ be a submatrix obtained by deleting $R_i$ and $C_j$. Then $i,j$th minor $M_{ij}$ is defined as follows:
		\[
			M_{ij}:= \det A_{ij}
		\]
		And $C_{ij}$, the $i,j$th cofactor of $A$ is defined as:
		\[
			C_{ij}:= (-1)^{i+j}M_{ij}
		\]
	\end{defn}

\end{frame}
%--- Next Frame ---%
\begin{frame}[t]{Determinant of $M_n$ Matrices}
	\begin{defn}
		[Determinant of $M_n$ Matrices]
		\[
			\det A := \sum_i^n a_{i \bar j}C_{i \bar j} = \sum_j^n a_{\bar i j}C_{\bar i j}
		\]
	\end{defn}
	\begin{block}
		{Calculation Procedure for General $A\in M_n$}
		\begin{enumerate}[STEP 1:]
			\item Select one $R_i$ or $C_j$
			\item Calculate $\det A$ from deleting each element in $R_i$ or $C_j$ from STEP 1
			\item For all $M_{ij}$ in STEP2, follow STEP1-2 recursively
		\end{enumerate}
	\end{block}
\end{frame}
%--- Next Frame ---%

\begin{frame}[t]{Example: $M_3$ Matrix}
	\begin{multline*}
		\det \begin{pmatrix}
			a &b & c \\
			d&e&f\\
			g&h&i
		\end{pmatrix}
		\\= a_{11} (-1)^{1+1}\left\vert \begin{matrix}
			e&f\\
			h&i
		\end{matrix}\right\vert + a_{12} (-1)^{1+2}\left\vert \begin{matrix}
			d&f\\
			g&i
		\end{matrix}\right\vert + a_{13} (-1)^{1+3}\left\vert \begin{matrix}
			d&e\\
			g&h
		\end{matrix}\right\vert
		\\= a(ei-fh)-b(di-fg)+c(dh-eg)
	\end{multline*}
	Important Note: Rule of Sarrus (\url{https://en.wikipedia.org/wiki/Rule_of_Sarrus}) only for $3\times 3$ matrices, NOT FOR $n\times n$ matrices!!!
\end{frame}
%--- Next Frame ---%
% subsection defining_the_determinant (end)
\subsection{Computing the Determinant} % (fold)
\label{sub:computing_the_determinant}
\begin{frame}[t]{Computing the Determinant of Special Form}
	In general, calculation of determinant is very complex except for some cases:
	\begin{thm}
		[9.1] If $A\in M_n$ is (1) lower-trangular, (2) upper-trangular, or (3) diagonal matrix, \[
			\det (A) = \prod_i^n A_{ii}
		\]
	\end{thm}
\end{frame}
%--- Next Frame ---%
\begin{frame}[t]{Determinant of $REFM$}
	\begin{thm}
		[9.2] Let $A_{REF}$ be the REFM of $A\in M_n$ by only using $ERO_1,ERO_2$. Then,
		\[
			\det A = \pm \det A_{REF}
		\] If only $ERO_2$ is used to make $A_{REF}$,
		\[
			\det A  = \det A_{REF}
		\]
	\end{thm}
	Proof Sketch: $\det(EM_1(R_i\leftrightarrow R_j))=-1, \det(EM_2(R_i\leftarrow R_i + kR_j))=1, \det(EM_3(R_i\leftarrow k R_i))=k$ and use Theorem 9.5(b)
	\begin{thm}
		[9.3: Main Property of Determinant] $A\in M_n$ is nonsingular iff $\det A \neq 0$
	\end{thm}
\end{frame}
%--- Next Frame ---%
% subsection computing_the_determinant (end)
% section the_determinant_of_a_matrix (end)

\section{Uses of The Determinant} % (fold)
\label{sec:uses_of_the_determinant}
\begin{frame}[t]{Determinant and Inverse Matrix}
	\begin{defn}
		[Adjoint] Let $C_{ij}\in\mathbb{R}$ be the $i,j$th cofactor of $A\in M_n$. Then, \[
			(adj A)_{ij}:=C_{ji} = (-1)^{j+i}M_{ji} \quad\forall i,j
		\]
	\end{defn}
	\begin{thm}
		[9.4 (a)]\[
			A^{-1} = \frac{1}{\det A}adj A
		\]
	\end{thm}
\end{frame}
%--- Next Frame ---%
\begin{frame}[t]{Cramer's rule}
	For any nonsingular matrix $A\in M_n$,
	\begin{thm}
		[9.4 (b):Cramer's Rule]
			The unique solution $\mathbf{x}=(x_1,\cdots,x_n)$ of $A\mathbf{x}=\mathbf{b}$ is:
			\[
				x_i = \frac{\det B_i}{\det A} \quad\forall i=1,\cdots,n
			\] When
			\begin{tiny}
				\begin{multline*}
					B_i := \begin{pmatrix}
						A_1 & \mathbf{b} & A_2
					\end{pmatrix},\\ A_1 : = \begin{pmatrix}
						a_{11}&a_{12}&\cdots&a_{1,i-1}\\
						a_{21}&a_{22}&\cdots&a_{2,i-1}\\
						\vdots\\
						a_{n1}&a_{n2}&\cdots&a_{n,i-1}
					\end{pmatrix}, A_2 : = \begin{pmatrix}
						a_{1,i+1}&a_{1,i+2}&\cdots&a_{1,n}\\
						a_{2,i+1}&a_{2,i+2}&\cdots&a_{2,n}\\
						\vdots\\
						a_{n,i+1}&a_{n,i+2}&\cdots&a_{n,n}
					\end{pmatrix}
				\end{multline*}
			\end{tiny}
	\end{thm}
\end{frame}
%--- Next Frame ---%
\begin{frame}[t]{Algebraic Properties of the Determinant Function}
	\begin{thm}
		[9.5] For $A\in M_n$,
		\begin{enumerate}[(a)]
			\item $\det A^T = \det A$
			\item $\det(AB)=\det A \det B$
			\item $\det(A+B)\neq \det A + \det B$ in general
		\end{enumerate}
	\end{thm}
	G-J Elimination ($\sim n^3/3$) is far, far, far more efficient than Cramer's rule ($\sim (n-1)((n+1)!)$) when $n$ is large
\end{frame}
%--- Next Frame ---%
% section uses_of_the_determinant (end)

\section{IS-LM Analysis via Cramer's Rule} % (fold)
\label{sec:is_lm_analysis_via_cramer_s_rule}
\begin{frame}[t]{IS-LM Analysis (Revisited)}

	\begin{align*}
		sY+ar &= I^0 + G\\
		mY-hr&=M_s-M^0
	\end{align*}
		Let's solve above IS-LM model:
	\[
		Y=\frac{\left\vert \begin{matrix}
			I^0+G&a\\
			M_s-M^0&-h
		\end{matrix}\right\vert}{\left\vert\begin{matrix}
			s&a\\
			m&-h
		\end{matrix}\right\vert}
,\quad
		r=\frac{\left\vert \begin{matrix}
			s&I^0+G\\
			m&M_s-M^0
		\end{matrix}\right\vert}{\left\vert\begin{matrix}
			s&a\\
			m&-h
		\end{matrix}\right\vert}
	\]
\end{frame}
%--- Next Frame ---%
\begin{frame}[t]{IS-LM Analysis}
	\begin{itemize}
		\item Expansionary fiscal policy: $\Delta G>0$
		\item Contractionary fiscal policy: $\Delta G<0$
		\item Expansionary monetary policy: $\Delta M_s>0$
		\item Contractionary monetary policy: $\Delta M_s<0$
	\end{itemize}
\end{frame}
%--- Next Frame ---%
% section is_lm_analysis_via_cramer_s_rule (end)
\end{document}
