%\documentclass[a4paper,11pt]{article}
%\usepackage[hyperref]{beamerarticle}

\documentclass[final]{beamer}

\usepackage[hangul]{kotex}
\usepackage{amsfonts,amsmath,xob-amssymb}

\usepackage{amsthm}
\newtheorem{defn}{Definition}
\newtheorem{thm}{Theorem}

\usepackage{cancel}
\usepackage{enumerate}

\mode<presentation>{
	\usetheme{Madrid}
	\usecolortheme{default}
	\usefonttheme{professionalfonts}
}

\def\b{\boldsymbol}

\mode<article>{
\usepackage{fullpage}
}
\usepackage{ulem}

\author[조남운]{\url{econMath.namun+2016sp@gmail.com}}
\title{One-Variable Calculus: Exponents and Logarithms}
\subtitle{CH5}

\begin{document}
	
\maketitle

\mode<presentation>{
\begin{frame}[t]{Table of Contents}
	\tableofcontents
\end{frame}
%--- Next Frame ---%
}

\section{Exponential Functions} % (fold)
\label{sec:exponential_functions}
\begin{frame}[t]{Exponential Functions}
	\begin{defn}
		[Exponential Function]
		$f:\mathbb{R}\rightarrow\mathbb{R}$ is \uline{exponential function} if $f(x)=\bar a \bar b^x,\quad \bar b>0$
	\end{defn}
	\begin{itemize}
		\item $x\in\mathbb{N}\Rightarrow f(x):=\bar a \prod_{i=1}^x \bar b$
		\item $f(0) := \bar a$
		\item $f(1/n):= \bar a \sqrt[n]{\bar b}$
		\item $f(m/n):= \bar a \sqrt[n]{\bar b^m}$
		\item $x<0 \Rightarrow f(x)=\bar a (1/\bar b)^{|x|}$
		\item Graph: convex, monotonic increasing ($b>1$) or decreasing ($b\in(0,1)$) function
	\end{itemize}
\end{frame}
%--- Next Frame ---%
% section exponential_functions (end)

\section{The Number $e$} % (fold)
\label{sec:the_number_e}
\begin{frame}[t]{Growth of an Account with Interest rate $r$}
	\begin{block}
		{Saving Account at $t=\bar T$ with Interest rate $\bar r$, Initial Endowment $\bar A$}
		\[
			A_t = \bar A \left(1+\bar r\right)^{\bar T}
		\]
	\end{block}
	\begin{block}
		{Compound Interest} If interest is compounded $n$ times per time unit, 
		\[
			A_t = \bar A \left(1+\frac{\bar r}{n}\right)^{n\bar T}
		\]
	\end{block}
	\begin{block}
		{Continuous Compounding} Compound Interest with $n\rightarrow \infty$
		\[
			A_t = \lim_{n\rightarrow \infty}\bar A \left(1+\frac{\bar r}{n}\right)^{n\bar T} = \bar A e^{\bar r\bar T}
		\]
	\end{block}
\end{frame}
%--- Next Frame ---%

\begin{frame}[t]{Number $e$}
	\begin{defn}
		[The Number $e$]
		\[
			e:=\lim_{n\rightarrow \infty}\left(1+\frac 1 n \right)^n = \sum_{n=0}^\infty \frac 1 {n!} \approx 2.718281693\cdots
		\]
	\end{defn}
	
	$e$ is irrational number.
	
	\begin{thm}
		[5.1]\[
			\lim_{n\rightarrow\infty}A\left(1+\frac r n \right)^{nt} = Ae^{rt}
		\]
	\end{thm}
	In general, an initial quantity $a_0$ with growth rate $r$ (per time unit) become $a_0 e^{rt}$ at time $t$ (time unit)
\end{frame}
%--- Next Frame ---%
% section the_number_e (end)

\section{Logarithms} % (fold)
\label{sec:logarithms}

\begin{frame}[t]{Logarithm}
	\begin{defn}
		[Base $b$ Logarithm]
		Base $b$ logarithm is an inverse of exponential function with base $b$
		\[
			f = b^x  \quad\Leftrightarrow\quad x = \log_b f
		\]
	\end{defn}

		\begin{itemize}
			\item $a^{\log_a z}=z$
			\item $\log_a{a^y}=y$
			\item Graph: concave, monotonic increasing ($b>1$) or convex, monotonic decreasing ($b\in(0,1)$)
		\end{itemize}
\end{frame}
%--- Next Frame ---%

\begin{frame}[t]{Natural Logarithm}
	\begin{defn}
		[Natural Logarithm]
		Base $e$ logarithm is \uline{natural logarithm}\[
			\ln x := \log_e x
		\]
	\end{defn}
	\[
		\ln x = y \quad \Leftrightarrow \quad e^y = x
	\]\[
		e^{\ln x} = x 
	\]\[
		\ln e^x = x
	\]
\end{frame}
%--- Next Frame ---%
% section logarithms (end)

	
\section{Properties of Exp and Log} % (fold)
\label{sec:properties_of_exp_and_log}
\begin{frame}[t]{Basic Properties of Exponential functions}
	$\forall r,s \in \mathbb{R}$,
	\begin{enumerate}
		\item $a^r a^s = a^{r+s}$
		\item $a^{-r} := 1/a^r$
		\item $a^r/a^s = a^{r-s}$
		\item $(a^r)^s=a^{rs}$
		\item $a^0 := 1$
	\end{enumerate}

\end{frame}


%--- Next Frame ---%

\begin{frame}[t]{Basic Properties of Logarithmic functions}
	$\forall r,s,a,b,c >0 \land a,c\neq 1,$
	\begin{enumerate}
		\item $\log(rs)=\log r + \log s$
		\item $\log(1/s)=-\log s$
		\item $\log(r/s) = \log r - \log s$
		\item $\log r^s = s\log r$
		\item $\log 1 = 0$
		\item $\log_a b = \frac{\log_c b }{\log_c a}= \frac{\ln b}{\ln a}$
	\end{enumerate}
	
(Ex5.4) Rule of 70 (or 69)
\end{frame}
%--- Next Frame ---%
% section properties_of_exp_and_log (end)	
	
\section{Derivatives of Exp and Log} % (fold)
\label{sec:derivatives_of_exp_and_log}
\begin{frame}[t]{Derivatives of Exp and Log functions}
	\begin{thm}
		[5.2]
		\[
			(e^x)^\prime = e^x
		\]\[
			(\ln x)^\prime = \frac 1 x 
		\]if $u\in \mathbf{C}^1,$ from chain rule,\[
			\left(e^{u}\right)^\prime = \left(e^{u}\right)u^\prime
		\]\[
			(\ln u )^\prime = \frac {u^\prime} {u }\tag{$u>0$}
		\]
	\end{thm}
\end{frame}
%--- Next Frame ---%
% section derivatives_of_exp_and_log (end)

\section{Applications} % (fold)
\label{sec:applications}
\begin{frame}[t]{Present Value}
	\begin{block}{Present Value (PV)}
		After time $T$, $A$ (at $t=0$) grow to $B$ (at $t=T$)
		\[
			B = Ae^{rT}
		\]
	
		$A$ is the present value (PV) of $B$ at $t=T$
	
		\[
			A = Be^{-rT}
		\]
	\end{block}
	
	\begin{itemize}
		\item PV of annuity
	\end{itemize}
\end{frame}
%--- Next Frame ---%

\begin{frame}[t]{Logarithmic Derivative}
	(Ex5.10) $(x^x)^\prime = ?$
	
	\begin{block}
		{Elasticity of $f$ is the Slope in log-log Graph of $f$}
		\[
			\epsilon := \frac{\frac{df}{f}}{\frac{dx}{x}} = \frac{d\ln f }{d\ln x}
		\]
	\end{block}
	
\end{frame}
%--- Next Frame ---%
% section applications (end)
	
\end{document}