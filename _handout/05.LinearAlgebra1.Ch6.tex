\documentclass[a4paper,11pt]{article}
\usepackage[hyperref]{beamerarticle}

% \documentclass[final]{beamer}

\usepackage[hangul]{kotex}
\usepackage{amsfonts,amsmath,xob-amssymb}

\usepackage{amsthm}
\newtheorem{defn}{Definition}
\newtheorem{thm}{Theorem}

\usepackage{cancel}
\usepackage{enumerate}

\mode<presentation>{
	\usetheme{Madrid}
	\usecolortheme{default}
	\usefonttheme{professionalfonts}
}

\def\b{\boldsymbol}

\mode<article>{
\usepackage{fullpage}
}
\usepackage{ulem}

\newcommand{\bb}{\mathbb}
\newcommand{\bd}{\mathbf}
\newcommand{\p}{\partial}

\newcommand{\mail}{\url{mailto:eyeofyou@korea.ac.kr}}

\author[조남운]{\mail}
\title{Linear Algebra}
\subtitle{CH6}

\begin{document}
	
\maketitle

\mode<presentation>{
\begin{frame}[t]{Table of Contents}
	\tableofcontents
\end{frame}
%--- Next Frame ---%
}

\section{Linear Systems} % (fold)
\label{sec:linear_systems}
\begin{frame}[t]{General Linear Equation}
	\begin{block}
		{General Linear Equation (multi variable)}
		\[
			\sum_{i=1}^n \bar a_i x_i = \bar a_1 x_1 + \bar a_2 x_2 + \cdots + \bar a_n x_n = \bar b
		\]
		More elegantly,
		\[
			\bar{\mathbf{a}}\cdot \mathbf{x} = \bar b ,\quad \bar{\mathbf{a}}=(\bar a_1 ,\cdots,\bar a_n), \mathbf{x}=(x_1,\cdots,x_n)
		\]
		Note: General Linear Equation (one variable) is:
		\[
			\bar a x = \bar b
		\]
	\end{block}
	\begin{itemize}
		\item $\bar a_i, \bar b$: parameters (given)
		\item $x_i$: variables
		\item Importance of linearity: Linear approximation
	\end{itemize}
\end{frame}

%--- Next Frame ---%
% section linear_systems (end)

\section{Examples of Linear Models} % (fold)
\label{sec:examples_of_linear_models}
\subsection{Markov Models of Unemployment} % (fold)
\label{sub:markov_models_of_unemployment}
\begin{frame}[t]{Ex3.Markov Models of Unemployment}
	\begin{block}
		{Main Concept and Assumptions}
		\begin{itemize}
			\item Binary state: employed ($E$) or unemployed ($U$)
			\item Four possible events with \uline{constant} probability: 
			\begin{itemize}
				\item $\Pr(E\rightarrow E)=\bar q$
				\item $\Pr(E\rightarrow U)=1-\bar q$
				\item $\Pr(U\rightarrow E)=\bar p$
				\item $\Pr(U\rightarrow U)=1-\bar p$
			\end{itemize}
			\item Markov process: Stochastic, memoryless process
			\item $x_t$: ratio of employed workers
			\item $y_t$: ratio of unemployed workers
		\end{itemize}
	\end{block}
\end{frame}
%--- Next Frame ---%
\begin{frame}[t]{Steady state of unemployment rate $y^\ast$}
	\begin{block}
		{Main Question}
		\begin{itemize}
			\item Existence of steady state s.t., 
			\[
				x_{t+1} \approx x_t = x^\ast,\quad y_{t+1} \approx y_t=y^\ast \quad \forall t>\bar T
			\]
			\item Stability of above steady state (--> need dynamic analysis: PASS)
		\end{itemize}
	\end{block}
	\begin{itemize}
		\item Hall (1966)'s estimation (by Races, male only)
		\begin{itemize}
			\item $p_{W}\approx 0.136$, $q_{W}\approx 0.998$
			\item $p_{B}\approx 0.102$, $q_{B}\approx 0.996$
		\end{itemize}
		\[
			y_{W}^\ast \approx 0.014 < \quad y_{B}^\ast \approx 0.037
		\]
	\end{itemize}
\end{frame}
%--- Next Frame ---%
% subsection markov_models_of_unemployment (end)
\subsection{IS-LM (Linear version)} % (fold)
\label{sub:is_lm_linear_version}
\begin{frame}[t]{Model}
	\begin{itemize}
		\item Hicks's interpretation of Keynes (1936)
	\end{itemize}
	\[
		Y=C+I+G \tag{IS Schedule: Real side}
	\]
	\begin{itemize}
		\item $C=\bar b Y, \quad \bar b \in (0,1)$
		\item $S=Y-C=(1-\bar b ) Y=\bar s Y$
		\item $I=\bar I^0-\bar a r$
	\end{itemize}
	\[
		\bar M_s = M_{dt}+M_{ds} = \underbrace{\bar m Y}_{M_{dt}} + \underbrace{\bar M^0 - \bar h r}_{M_{ds}} \tag{LM Schedule: Monetary side}
	\]
	\begin{itemize}
		\item Endogenous variables: $r,Y$
		\item Exogenous variables (given parameters): $M_s, G, a, h, I^0, M^0, m, s, b$
	\end{itemize}
\end{frame}
%--- Next Frame ---%
% subsection is_lm_linear_version (end)
% section examples_of_linear_models (end)
	
\end{document}
