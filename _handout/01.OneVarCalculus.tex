\documentclass[a4paper,11pt]{article}
\usepackage[hyperref]{beamerarticle}

% \documentclass[final]{beamer}

\usepackage[hangul]{kotex}
\usepackage{amsfonts,amsmath,xob-amssymb}

\usepackage{amsthm}
\newtheorem{defn}{Definition}
\newtheorem{thm}{Theorem}

\usepackage{cancel}

\mode<presentation>{
	\usetheme{Madrid}
	\usecolortheme{default}
	\usefonttheme{professionalfonts}
}

\def\b{\boldsymbol}

\mode<article>{
\usepackage{fullpage}
}
\usepackage{ulem}

\newcommand{\bb}{\mathbb}
\newcommand{\bd}{\mathbf}
\newcommand{\p}{\partial}

\newcommand{\mail}{\url{mailto:econMath.namun+2016f@gmail.com} (한신대)\\\url{mailto:ku.econ205+2016f@gmail.com} (고려대)}

\author[조남운]{\mail}
\title{One-Var Calculus}
\subtitle{CH2}

\begin{document}

	\maketitle

\mode<presentation>{
\begin{frame}[t]{Table of Contents}
	\tableofcontents
\end{frame}
}
%--- Next Frame ---%

\section{Functions on $\mathbb{R}$} % (fold)
\begin{frame}[t]{Functions}

	\begin{defn}[Polynomial]
		\uline{Polynomial} is sum of monomials
	\end{defn}
	
	\begin{defn}[Monomial]
		$f$ is \uline{monomial} if:
			\[
				f(x) = \bar a x^{\bar k},\quad \bar a \in \mathbb{R}, \bar k\in \mathbb{N}+\{0\}
			\]
	\end{defn}

	\begin{defn}
		[Rational function]
		\[
			P(x)/Q(x),\quad \text{$P,Q \in$ set of polynomials} \land Q\neq 0
		\]
	\end{defn}
	
	\begin{defn}
		[Exponential function]
		$f$ is \uline{exponential function} if: $
			f(x) = \bar a {\bar b}^x,\quad \bar a, \bar b \in \mathbb{R}
		$
	\end{defn}

\end{frame}
%--- Next Frame ---%

\begin{frame}[t]{Increasing, Decreasing}
	\begin{defn}
		[Increasing Function]
		$f$ is \uline{increasing} if:\[
			\forall x_1,x_2, \quad x_1 >x_2 \Rightarrow f(x_1)>f(x_2)
		\]
	\end{defn}
	\begin{defn}
		[Decreasing Function]
		$f$ is \uline{decreasing} if:\[
			\forall x_1,x_2, \quad x_1 >x_2 \Rightarrow f(x_1)<f(x_2)
		\]
	\end{defn}
\end{frame}
%--- Next Frame ---%

\begin{frame}[t]{Minimum, Maximum}
	\begin{defn}
		[Local (Relative) minimum]
		$(x_0,f(x_0))$ is \uline{local minimum} of $f$ if $f$ changes from decreasing to increasing
	\end{defn}
	
	\begin{defn}
		[Local (Relative) maximum]
		$(x_0,f(x_0))$ is \uline{local maximum} of $f$ if $f$ changes from increasing to decreasing
	\end{defn}
	
	Note: Boundary max/min
	
	\begin{defn}
		[Global (Absolute) minimum]
		$(x_0,f(x_0))$ is \uline{Global minimum} of $f:D\rightarrow \mathbb{R}$ if \[
			f(x_0)\le f(x) \quad \forall x \in D
		\]
	\end{defn}
\end{frame}
%--- Next Frame ---%

\begin{frame}[t]{Interval}
	\begin{defn}
		[Open Interval]
		\[
			(\bar a,\bar b):=\{x\in \mathbb{R}\vert \bar a < x < \bar b \}
		\]
	\end{defn}
	
	\begin{defn}
		[Closed Interval]
		\[
			[\bar a,\bar b]:=\{x\in \mathbb{R}\vert \bar a \le x \le  \bar b \}
		\]
	\end{defn}
	
	\begin{itemize}
		\item half-open (or half-closed) intervals
		\[
			(\bar a, \bar b],\quad [\bar a, \bar b),\cdots
		\]
		\item infinite intervals
		\[
			(-\infty, \bar a],\quad (\bar a, \infty),\cdots
		\]
	\end{itemize}
\end{frame}
%--- Next Frame ---%

\section{Linear Functions} % (fold)
\label{sec:linear_functions}

% section linear_functions (end)
\begin{frame}[t]{Linear Function}
	\begin{defn}
		[Linear Function]
		Polynomial of degree 0 (Constant function) or 1
	\end{defn}
	\begin{itemize}
		\item Properties
		\begin{itemize}
			\item Graph: Straight line
			\item Constant slope 
			\begin{defn}
				[Slope of Linear function $f$]
				\[
					Slope:= \frac{f(x_1)-f(x_0)}{x_1-x_0}
				\]
			\end{defn}
		\end{itemize}
	\end{itemize}
	\begin{thm}
		[2.1]
		The line (Graph of $f$) with slope $\bar m$ and vertical intercept $\bar b$ $\Rightarrow$ $f(x)=\bar m x + \bar b$
	\end{thm}
\end{frame}
%--- Next Frame ---%

\section{The Slope of Nonlinear Functions} % (fold)
\label{sec:the_slope_of_nonlinear_functions}

% section the_slope_of_nonlinear_functions (end)
\begin{frame}[t]{Nonlinear function}
	\begin{defn}
		[Nonlinear function]
		$f$ is nonlinear function if $f$ is not linear function
	\end{defn}
	
	\begin{defn}
		[Derivative at $(\bar x_0, f(\bar x_0))$ ]
		$f^\prime(\bar x_0)$ := \uline{Derivative} of $f$ at $(\bar x_0, f(\bar x_0))$ is the slope of the tangent line to the graph of $f$ at $(\bar x_0, f(\bar x_0))$. \textit{i.e.,}
		\[
			f^\prime(\bar x_0):= \lim_{h\rightarrow 0}\frac{f(\bar x_0+h)-f(\bar x_0)}{h}
		\]
	\end{defn}
	\begin{itemize}
		\item Alternative Notations
		\[
			f^\prime(x_0) \equiv \frac{df}{dx}(x_0) \equiv \left.\frac{df}{dx}\right\vert_{x=x_0}
		\]
	\end{itemize}
\end{frame}
%--- Next Frame ---%

\section{Computing Derivatives} % (fold)
\label{sec:computing_derivatives}

\begin{frame}[t]{Computing Derivatives}
	\begin{thm}
		[2.2]
		\[
			f(x)=x^2 \Rightarrow f^\prime (x) = 2x
		\]
	\end{thm}

	\begin{thm}
		[2.3]
		\[
			f(x)=x^{\bar k} , \bar k \in \mathbb{N} \Rightarrow f^\prime (x) = \bar k x^{\bar k - 1}
		\]
	\end{thm}
\end{frame}
%--- Next Frame ---%

\begin{frame}[t]{Operation of Functions}
	\begin{defn}
		[Operation of Functions]
		\[
			(f\pm g)(x) := f(x)\pm g(x)
		\]
		\[
			(f\cdot g)(x) := f(x)g(x)
		\]\[
			(f/g)(x) := f(x)/g(x)
		\]
	\end{defn}
\end{frame}
%--- Next Frame ---%

\begin{frame}[t]{Rules for Computing Derivatives}
	\begin{thm}
		[2.4]
		\[
			(f\pm g)\prime = f^\prime \pm g^\prime
		\]\[
			(\bar k f)^\prime = \bar k (f^\prime)
		\]\[
			(f\cdot g)^\prime = f^\prime g + f g^\prime 
		\]\[
			\left(\frac{f}{g}\right)^\prime = \frac{f^\prime g - fg^\prime}{g^2}
		\]\[
			(f^{\bar n})^\prime = \bar n f^{\bar n - 1} f^\prime 
		\]
	\end{thm}
\end{frame}
%--- Next Frame ---%

% section computing_derivatives (end)

\section{Differentiabillty and Continuity} % (fold)
\label{sec:differentiablilty_and_continuity}
\begin{frame}[t]{Differentiable}
	\begin{defn}
		[Differentiable]
		$f:D\rightarrow \mathbb{R}$ is \uline{differentiable} if \[
			\forall x_0\in D\text{ and }\forall\{h_n\}\rightarrow 0, \exists \text{unique } f^\prime(x_0) = \lim_{h_n\rightarrow 0}\frac{f(x_0+h_n)-f(x_0)}{h_n}
		\]
	\end{defn}
	\begin{itemize}
		\item Geometrical meaning: graph of $f$ is smooth
		\item $f\in \mathbf{C}^1$
	\end{itemize}
\end{frame}
%--- Next Frame ---%

\begin{frame}[t]{Continuous}
	\begin{defn}
		[Continuous]
		$f:D\rightarrow \mathbb{R}$ is \uline{continuous} if:\[
			\forall x_0 \in D,\quad x_n \rightarrow x_0 \Rightarrow f(x_n)\rightarrow f(x_0)
		\]
	\end{defn}
	\begin{itemize}
		\item Geometrical meaning: graph of $f$ is not disconnected
	\end{itemize}
\end{frame}
%--- Next Frame ---%

% section differentiablilty_and_continuity (end)

\section{Higher-order Derivatives} % (fold)
\label{sec:higher_order_derivatives}

\begin{frame}[t]{Second Derivative of $f$}
	\begin{defn}
		[Second Derivative of $f\in \mathbf{C}^2$]
		\[
			f^{\prime\prime} := (f^\prime)^\prime \equiv \frac{d}{dx}\left(\frac{df}{dx}\right) \equiv \frac{d^2f}{dx^2}
		\]
	\end{defn}
	\begin{itemize}
		\item $\mathbf{C}^3$
		\[
			f^{\prime\prime\prime},\quad \frac{d^3f}{dx^3}
		\]
		\item $\mathbf{C}^k$
		\[
			f^{(k)},\quad \frac{d^kf}{dx^k}
		\]
		\item Polynomial is $\mathbf{C}^\infty$
	\end{itemize}
\end{frame}
%--- Next Frame ---%
% section higher_order_derivatives (end)

\section{Approximation by Differentials} % (fold)
\label{sec:approximation_by_differentials}


\begin{frame}[t]{Approximation}
	\begin{itemize}
		\item $\Delta x$: Change in $x$ (general representation)
		\item $dx$: ``Small'' change in $x$ (or $\Delta x$ which is sufficiently close to 0)
		\item Suppose $x_0$ is changed to $x_0+h$, and $h$ is sufficiently close to 0. then, 
	\end{itemize}
	\[
		\frac{f(x_0+h)-f(x_0)}{h}\equiv \frac{\Delta f}{\Delta x} \approx f^\prime (x_0)
	\]
	\[
		\Delta f \approx f^\prime (x_0) \Delta x
	\] or, \[
		df = f^\prime (x_0) dx
	\]
\end{frame}
%--- Next Frame ---%
% section approximation_by_differentials (end)

\end{document}
