\documentclass[a4paper,11pt]{article}
\usepackage[hyperref]{beamerarticle}

% \documentclass[final]{beamer}

\usepackage{kotex}
\usepackage{amsfonts,amsmath,xob-amssymb}

\usepackage{amsthm}
\newtheorem{defn}{Definition}
\newtheorem{thm}{Theorem}

\usepackage{cancel}
\usepackage{enumerate}

\mode<presentation>{
	\usetheme{Madrid}
	\usecolortheme{default}
	\usefonttheme{professionalfonts}
}

\def\b{\boldsymbol}

\mode<article>{
\usepackage{fullpage}
}
\usepackage{ulem}

\newcommand{\bb}{\mathbb}
\newcommand{\bd}{\mathbf}
\newcommand{\p}{\partial}

\newcommand{\mail}{\url{econMath.namun+2016su@gmail.com}}

\author[조남운]{\mail}
\title{Unconstrained Optimization}
\subtitle{Ch.17}

\begin{document}
	
\maketitle

\mode<presentation>{
\begin{frame}[t]{Table of Contents}
	\tableofcontents
\end{frame}
%--- Next Frame ---%
}

\section{Definitions} % (fold)
\label{sec:definitions}
\begin{frame}[t]{Definitions}
	\begin{defn}
		[(strict) max/min, (strict) local max/min]
		Let $f:U\in \bb{R}^n\rightarrow \bb{R}$
		\begin{enumerate}
			\item A point $\bd{x}^\ast$ is a \uline{(global, or absolute) max, maximizer, maximum point} of $f$ on $U$ if $f(\bd{x^\ast})\ge f(\bd{x})\quad\forall \bd{x}\in U$
			\item $\bd{x^\ast}\in U$ is a \uline{strict (global, or absolute) max} if $\bd{x^\ast}$ is a max and $f(\bd{x^\ast})>f(\bd{x})\quad\forall \bd{x}\in U-\{\bd{x^\ast}\}$
			\item $\bd{x^\ast}\in U$ is a \uline{local (relative) max} of $f$ if $\exists \epsilon>0 \quad s.t.$ $f(\bd{x^\ast})\ge f(\bd{x})\quad\forall \bd{x}\in B_\epsilon(\bd{x^\ast})\cap U$
			\item $\bd{x^\ast}\in U$ is a \uline{strict local (relative) max} of $f$ if $\exists \epsilon>0 \quad s.t.$ $f(\bd{x^\ast})> f(\bd{x})\quad\forall \bd{x}\in B_\epsilon(\bd{x^\ast})\cap U$
		\end{enumerate}
	\end{defn}
	\begin{itemize}
		\item Definition of min: $>,\ge \rightarrow <, \le$ 
	\end{itemize}
\end{frame}
%--- Next Frame ---%
% section definitions (end)

\section{First Order Conditions} % (fold)
\label{sec:first_order_conditions}
\begin{frame}[t]{FOC}
	\begin{thm}
		[17.1] Let $f:U\in\bb{R}^n\rightarrow\bb{R}$ be a $C^1$ function. If $\bd{x^\ast}$ is a local max or min of $f$ and $\bd{x^\ast}$ is an interior point of $U$, then\[
			\frac{\p f}{\p x_i}(\bd{x^\ast})=0\quad\forall i
		\] In short, \[
			Df_{\bd{x}}(\bd{x^\ast}) = \bd{0}
		\] $\bd{x^\ast}$ is a critical point of $f$
	\end{thm}
	Note: Compare with one-var version FOC (Theorem 3.3)	\begin{thm}
		[3.3: First Order Condition (FOC)]
		$x_0$ is an interior max or min of $f$ $\Rightarrow$ $x_0$ is a critical point of $f$. $i.e.,$ $f^\prime(x_0)=0$
		(Inverse is not always true)
	\end{thm}
\end{frame}
%--- Next Frame ---%
% section first_order_conditions (end)

\section{Second Order Conditions} % (fold)
\label{sec:second_order_conditions}
\begin{frame}[t]{SOC (Sufficient Conditions)}
	\begin{thm}
		[17.2] Let $f:U\in\bb{R}^n\rightarrow\bb{R}$ be a $C^2$ function and $U$ is open. Suppose $\bd{x^\ast}$ is a critical point of $f$. ($i.e.,Df_{\bd{x}}(\bd{x^\ast})=\bd{0}$) Then, 
		\begin{enumerate}
			\item If Hessian ($D^2 f_\bd{x}(\bd{x^\ast})$) is ND, then $\bd{x}^\ast$ is a strict local max of $f$
			\item If Hessian ($D^2 f_\bd{x}(\bd{x^\ast})$) is PD, then $\bd{x}^\ast$ is a strict local min of $f$
			\item If Hessian is ID, $\bd{x^\ast}$ is neither a local max nor local min of $f$. (saddle point)
		\end{enumerate}
	\end{thm}
	Note: one-var version: (Theorem 3.4)\[
		f^\prime(x^\ast)=0\quad\land\quad f^{\prime\prime}<0 \quad\Rightarrow\quad x^\ast\text{ is a local max}
	\]
\end{frame}
%--- Next Frame ---%

\begin{frame}[t]{SOC (Necessary Conditions)}
	\begin{thm}
		[17.6] Let $f:U\in\bb{R}^n\rightarrow\bb{R}$ be a $C^2$ function and $U$ is open. Then,
		\begin{enumerate}
			\item $\bd{x^\ast}$ is a local min of $f$ $\Rightarrow$ $Df(\bd{x^\ast})=\bd{0}\quad\land\quad D^2f(\bd{x^\ast})$ is PSD
			\item $\bd{x^\ast}$ is a local max of $f$ $\Rightarrow$ $Df(\bd{x^\ast})=\bd{0}\quad\land\quad D^2f(\bd{x^\ast})$ is NSD
		\end{enumerate}
	\end{thm}
	Note: one-var version: \[
		x^\ast\text{ is local max} \quad\Rightarrow\quad x^\prime=0\quad\land\quad f^{\prime\prime}\le 0
	\]
\end{frame}
%--- Next Frame ---%
% section second_order_conditions (end)

\section{Global Maxima and Minima} % (fold)
\label{sec:global_maxima_and_minima}
\begin{frame}[t]{Finding Global Max/Min}
	Different from one-var function, condition 1 (below) is not true when $f$ is multi-var function
	\begin{block}
		{Sufficient Conditions for Global Max/Min ($f:I\in\bb{R}\rightarrow\bb{R}$)}
		\begin{enumerate}
			\item $x^\ast$ is a local max/min and $x^\ast$ is the only critical point of $f$ in $I$
			\item $f^{\prime\prime}\le 0\quad\forall I$. $i.e.,$ $f$ is concave on $I$ (max)
			\begin{itemize}
				\item $f^{\prime\prime}\le 0 \quad \forall I$ (max)
				\item $f^{\prime\prime}\ge 0 \quad \forall I$ (min)
			\end{itemize}
		\end{enumerate}
	\end{block}
	However, condition 2 is true even when $f$ is multi-var function!
	\begin{thm}
		[17.8]
		Let $f:U\in \bb{R}^n\rightarrow \bb{R}$ be a $C^2$ function with convex open domain $U$.
		\begin{enumerate}
			\item $DF(\bd{x^\ast})=\bd{0}$ and $D^2f_{\bd{x}}$ is PSD on $U$ $\Rightarrow $ $\bd{x^\ast}$ is a global max of $f$ on $U$
			\item $DF(\bd{x^\ast})=\bd{0}$ and $D^2f_{\bd{x}}$ is NSD on $U$ $\Rightarrow $ $\bd{x^\ast}$ is a global min of $f$ on $U$
		\end{enumerate}
	\end{thm}
\end{frame}
%--- Next Frame ---%
% section global_maxima_and_minima (end)

\section{Economic Applications} % (fold)
\label{sec:economic_applications}
\begin{frame}[t]{Ordinary Least Squares (OLS)}
	\begin{block}
		{OLS}
		Find $y=\bd{x}\bd{\beta}+{c}$ for given $N$ data $X = \begin{pmatrix}
			\bd{x_1}\\
			\vdots\\
			\bd{x_N}
		\end{pmatrix}$, $Y=\begin{pmatrix}
			y_1\\
			\vdots\\
			y_N
		\end{pmatrix}$ satisfying:
		\[
			\arg\min_{\bd{\beta},{c}} \sum_i^N ((\bd{x_i}\bd{\beta}+{c})-y_i)^2 \tag{Least Square}
		\]\[
			y=\bd{x}\bd{\beta}+{c} = x_1 \beta_1 + \cdots + x_m \beta_m + {c}
		\]% given data are:\[
% 			(y_1, \bd{x_1}) = (y_1, x_{11},x_{12},\cdots,x_{1m}), (y_2,\bd{x_2}), \cdots , (y_N,\bd{x_N})
% 		\]
	\end{block}
	Note: given points -- $(y_1, \bd{x_1}) = (y_1, x_{11},x_{12},\cdots,x_{1m})$, $(y_2,\bd{x_2}), \cdots , (y_N,\bd{x_N})$ -- are not variables. Our object is to find $\bd{\beta^\ast}, {c^\ast}$ (linear equation) from given data $X,Y$.
\end{frame}
%--- Next Frame ---%

\begin{frame}[t]{OLS: Solution}
	Let our object function $f(\beta_1,\cdots,\beta_n,c):=\sum_i^N ( (\bd{x_i}\bd{\beta}+{c})-y_i)^2$. Then FOC is:
	\[
		Df_{\bd{\beta},c}(\bd{\beta}^\ast,c^\ast)=\bd{0} \tag{FOC}
	\]
	This leads to $m+1$ equations:\[
		\frac{\p f}{\p \beta_1}(\bd{\beta^\ast},c^\ast) = 2(\bd{x_1}\bd{\beta^\ast}+c^\ast-y_1)x_{11}+2(\bd{x_2}\bd{\beta^\ast}+c^\ast-y_2)x_{21}+\cdots+2(\bd{x_N}\bd{\beta^\ast}+c^\ast-y_N)x_{N1} = 0
	\]\[
		\cdots
	\]\[
		\frac{\p f}{\p \beta_m}(\bd{\beta^\ast},c^\ast) = 2(\bd{x_1}\bd{\beta^\ast}+c^\ast-y_1)x_{1m}+2(\bd{x_2}\bd{\beta^\ast}+c^\ast-y_2)x_{2m}+\cdots+2(\bd{x_N}\bd{\beta^\ast}+c^\ast-y_N)x_{Nm} = 0
	\]
	\[
			\Rightarrow 2(\bd{x_1}\bd{\beta^\ast}+c^\ast-y_1)\bd{x_1}^T+2(\bd{x_2}\bd{\beta^\ast}+c^\ast-y_2)\bd{x_2}^T+\cdots+2(\bd{x_N}\bd{\beta^\ast}+c^\ast-y_N)\bd{x}_{N}^T = \bd{0}_{m\times 1}
		\tag{B}\]
	\[
		\frac{\p f}{\p c}(\bd{\beta^\ast},c^\ast) = 2(\bd{x_1}\bd{\beta^\ast}+c^\ast-y_1)1+2(\bd{x_2}\bd{\beta^\ast}+c^\ast-y_2)1+\cdots+2(\bd{x_N}\bd{\beta^\ast}+c^\ast-y_N)1 = 0 \tag{C}
	\]
\end{frame}
%--- Next Frame ---%
\begin{frame}[t]{OLS (2)}
	\[
		Remember \quad X = \begin{pmatrix}
			x_{11} & x_{12} & \cdots & x_{1m}\\
			x_{21} & x_{22} & \cdots & x_{2m}\\
			\vdots & \vdots & \vdots & \vdots\\
			x_{N1} & x_{N2} & \cdots & x_{Nm}
		\end{pmatrix}  = \begin{pmatrix}
			C_1 & \cdots & C_m
		\end{pmatrix}=\begin{pmatrix}
			\bd{x}_1 \\ \bd{x}_2 \\ \vdots \\ \bd{x}_N
		\end{pmatrix}
	\]
	Rearrange FOCs:% $\bd{\beta^\ast}$: \[
% 		\begin{pmatrix}
% 			\bd{x}_1^T\bd{\beta^\ast}+c-Y_1 & \cdots & \bd{x}_N^T\bd{\beta^\ast}+c-Y_N
% 		\end{pmatrix}\begin{pmatrix}
% 		\bd{x}_1^T\\\vdots\\\bd{x}_N^T
% 		\end{pmatrix}=\bd{0}_{m\times 1}
% 	\]
	\[
		\begin{pmatrix}
			\bd{x_1}^T & \bd{x_2}^T & \cdots & \bd{x_N}^T
		\end{pmatrix}\begin{pmatrix}
			\bd{x_1}\bd{\beta^\ast}+c^\ast-y_1 \\\vdots\\\bd{x_N}\bd{\beta^\ast}+c^\ast-y_N
		\end{pmatrix} = X^T (X\bd{\beta^\ast}+\bd{1}_{N\times 1}c^\ast-Y)=\bd{0}_{m\times 1} \tag{B2}
	\]\[
		c^\ast = \frac{1}{N}\begin{pmatrix}
			1 & 1 & \cdots & 1
		\end{pmatrix}\begin{pmatrix}
			y_1 - \bd{x_1}\bd{\beta^\ast} \\\vdots\\y_N - \bd{x_N}\bd{\beta^\ast}
		\end{pmatrix}=\frac{1}{N}\bd{1}_{1\times N}(Y-X\bd{\beta^\ast})\tag{C2}
	\]
\end{frame}
%--- Next Frame ---%
\begin{frame}[t]{OLS (3)}
	From (C2) and (B2),\[
		X^T\left(X\bd{\beta^\ast}+\bd{1}_{N\times 1}\frac{1}{N}\bd{1}_{1\times N}(Y-X\bd{\beta^\ast})-Y\right) = \bd{0}_{m\times 1}\tag{D}
	\]
	Rearrange (D) with regard to $\bd{\beta^\ast}$ yields:
	\[
		X^T(X - \frac{1}{N}\bd{1}_{N\times 1}\bd{1}_{1\times N}X)\bd{\beta^\ast}= X^T(I_N-\frac{1}{N}\bd{1}_{N\times 1}\bd{1}_{1\times N})Y
	\]
	Let $\bd{1}_{N\times 1}\bd{1}_{1\times N} = \bd{1}_N$. ($N\times N$ matrix with all elements are $1$)
	\[
		\bd{\beta^\ast} = \left(X^T \left(X-\frac{1}{N}\bd{1}_{N} X\right)\right)^{-1}\left(X^T \left(Y - \frac{1}{N}\bd{1}_N Y\right)\right)
	\]\[
		=\left(X^T\left(I_N - \frac{1}{N}\bd{1}_N\right)X\right)^{-1}\left(X^T\left(I_N - \frac{1}{N}\bd{1}_N\right)Y\right)
	\]
\end{frame}
%--- Next Frame ---%
\begin{frame}[t]{OLS (4)}
	\begin{block}
		{Sample Mean $\bar X, \bar Y$}
		\[
			\frac{1}{N}\bd{1}_N X = \frac{1}{N}\begin{pmatrix}
				1  & \cdots & 1\\
				1  & \cdots & 1\\
				\vdots & \vdots & \vdots\\
				1 & \cdots & 1
			\end{pmatrix}\begin{pmatrix}
			x_{11} & \cdots & x_{1m}\\
			x_{21} & \cdots & x_{2m}\\
			\vdots & \vdots & \vdots\\
			x_{N1} & \cdots & x_{Nm}
		\end{pmatrix} = \begin{pmatrix}
			\bar{\bd{x}}_1 & \cdots & \bar{\bd{x}}_m\\
			\bar{\bd{x}}_1 & \cdots & \bar{\bd{x}}_m\\
			\cdots\\
			\bar{\bd{x}}_1 & \cdots & \bar{\bd{x}}_m\\
		\end{pmatrix}= \bar{X}
		\]
		\[
			\frac{1}{N}\bd{1}_N Y = \frac{1}{N}\begin{pmatrix}
				1 & 1 & \cdots & 1\\
				\vdots & \vdots & \vdots & \vdots\\
				1 & 1 & \cdots & 1
			\end{pmatrix}\begin{pmatrix}
			y_1 \\
			\vdots\\
			y_N
		\end{pmatrix} = \begin{pmatrix}
			\bar{y}\\
			\vdots\\
			\bar{y}
		\end{pmatrix}=\bar{Y}
		\]
		Here $\bar{\bd{x}}_j$, $\bar{y}$ means sample mean of $x_{ij}$, $y_i$\[
			\bar{\bd{x}}_j := \frac{1}{N}\sum_i^N x_{ij}, \quad \bar{y}:=\frac{1}{N}\sum_i^N {y_i}
		\]
	\end{block}
\end{frame}
%--- Next Frame ---%

\begin{frame}[t]{OLS (5)}
	Therefore, $\bd{\beta^\ast}$ is:
	\[
		\bd{\beta^\ast} = (X^T(X-\bar X))^{-1}X^T(Y-\bar Y)
	\] Note1: If $N\rightarrow \infty$, then $I_N - \frac{1}{N}\bd{1}_N \rightarrow I_N$ and\[
			\beta^\ast \rightarrow (X^T X)^{-1}X^T Y
		\]
		% This implies that $c$ becomes trivial as size of data ($N$) increasing
		
		Note2: We should check SOC: whether $H=D^2f_{\bd{\beta},c}(\bd{\beta^\ast},c^\ast)$ is PD or not.  Our object function has quadratic form with positive sign with regard to $\bd{\beta},c$ \uline{when $\bd{x_j}$ is independent with each other} and this means $f$ is PD (when $\bd{x_j}$ is independent with each other: covariance with other variables are 0).
		
		Note3: Some researchers denote $X^T$ by $X^\prime$
\end{frame}
%--- Next Frame ---%

% section economic_applications (end)

\end{document}

