\documentclass[a4paper,11pt]{article}
\usepackage[hyperref]{beamerarticle}

% \documentclass[final]{beamer}

\usepackage[hangul]{kotex}
\usepackage{amsfonts,amsmath,xob-amssymb}

\usepackage{amsthm}
\newtheorem{defn}{Definition}
\newtheorem{thm}{Theorem}

\usepackage{cancel}
\usepackage{enumerate}

\mode<presentation>{
	\usetheme{Madrid}
	\usecolortheme{default}
	\usefonttheme{professionalfonts}
}

\def\b{\boldsymbol}

\mode<article>{
\usepackage{fullpage}
}
\usepackage{ulem}

\newcommand{\bb}{\mathbb}
\newcommand{\bd}{\mathbf}
\newcommand{\p}{\partial}

\newcommand{\mail}{\url{mailto:eyeofyou@korea.ac.kr}}

\author[조남운]{\mail}
\title{One-Variable Calculus: Applications}
\subtitle{CH3}

\begin{document}

	\maketitle

\mode<presentation>{
\begin{frame}[t]{Table of Contents}
	\tableofcontents
\end{frame}
}
%--- Next Frame ---%

\section{Using the First Derivatives for Graphing} % (fold)
\label{sec:using_the_first_derivative_for_graphing}

\begin{frame}[t]{Positive Derivative implies Increasing Function}
	Suppose $f$ is continuously differentiable at $x_0$ ($i.e.,$ $\exists f^\prime$ $\land$ $f^\prime$ is continuous)
	\begin{thm}[3.1]
		\begin{enumerate}
			\item $f^\prime (x_0)>0 \Rightarrow \exists \bar\alpha,\bar\beta\in\mathbb{R} $ s.t. $x_0\in(\bar\alpha,\bar\beta)\land f$ is increasing on $(\bar\alpha,\bar\beta)$
			\item $f^\prime (x_0)<0 \Rightarrow \exists \bar\alpha,\bar\beta\in\mathbb{R} $ s.t. $x_0\in(\bar\alpha,\bar\beta)\land f$ is decreasing on $(\bar\alpha,\bar\beta)$
		\end{enumerate}
	\end{thm}
	\begin{thm}
		[3.2]
		\begin{enumerate}
			\item $f^\prime>0$ on $(\bar a, \bar b)\subset D \Rightarrow f$ is increasing on $(\bar a,\bar b)$
			\item $f^\prime<0$ on $(\bar a, \bar b)\subset D \Rightarrow f$ is decreasing on $(\bar a,\bar b)$
			\item $f$ is increasing on $(\bar a, \bar b) \Rightarrow f^\prime \ge 0$ on $(\bar a, \bar b)$
			\item $f$ is decreasing on $(\bar a, \bar b) \Rightarrow f^\prime \le 0$ on $(\bar a, \bar b)$
		\end{enumerate}
	\end{thm}
\end{frame}
%--- Next Frame ---%

\begin{frame}[t]{Graph Sketching using First Derivatives}
	\begin{block}{Procedure}
		\begin{enumerate}[(STEP 1)]
			\item Find all $x_i^\ast$ s.t. $f^\prime (x_i^\ast)=0$ (critical points), boundary points, and around undefined points
			\item Calculate $f(x_i^\ast)$
			\item Make table for graph sketch
			\begin{itemize}
				\item $f^\prime >0 \Rightarrow \nearrow$
				\item $f^\prime <0 \Rightarrow \searrow$
			\end{itemize}
		\end{enumerate}
	\end{block}
	\[
		f(x) = x^4 -8x^3 + 18x^2 - 11 \tag{Ex3.1}
	\]
\end{frame}
%--- Next Frame ---%

% section using_the_first_derivative_for_graphing (end)

\section{Second Derivatives and Convexity} % (fold)
\label{sec:second_derivatives}

\begin{frame}[t]{Convexity and Concavity}
	\begin{defn}
		[Convex (Concave up), Concave (Concave down)]
		$f$ is convex on $(\bar \alpha, \bar \beta)$ iff:
		\[
			f\left((1-t)\bar a + t \bar b \right)\le (1-t)f(\bar a)+ tf(\bar b),\quad \forall t\in [0,1] \quad\forall \bar a, \bar b \in [\alpha,\beta]
		\]
		$f$ is concave on $(\bar \alpha, \bar \beta)$ iff:
		\[
			f\left((1-t)\bar a + t \bar b \right) \ge (1-t)f(\bar a)+ tf(\bar b),\quad \forall t\in [0,1] \quad\forall \bar a, \bar b \in [\alpha,\beta]
		\]
	\end{defn}
	\begin{center}
		\begin{tabular}{c|c|c}
		&	$f^\prime >0 $& $f^\prime<0$	\\
		\hline\hline
		$f^{\prime\prime}>0$&	&\\
		\hline
		$f^{\prime\prime}<0$&&\\
		\end{tabular}
	\end{center}
\end{frame}
%--- Next Frame ---%


\begin{frame}[t]{Using Second Derivative for Graph Sketch}
	\begin{block}{Procedure}
		\begin{enumerate}[(STEP 1)]
			\item Find all $x_i^\ast$ s.t. $f^\prime (x_i^\ast)=0$ (critical points), \uline{$f^{\prime\prime}(x_i^\ast)=0$}, boundary points, and around undefined points
			\item Calculate $f(x_i^\ast)$
			\item Make table for graph sketch
		\end{enumerate}
	\end{block}
	\[
		f(x) = x^4 -8x^3 + 18x^2 - 11 \tag{Ex3.1}
	\]
\end{frame}
%--- Next Frame ---%

% section second_derivatives (end)

\section{Graphing Rational Functions} % (fold)
\label{sec:graphing_rational_functions}
\begin{frame}[t]{Graphing Rational Function}
	\begin{block}{Procedure}
		\begin{enumerate}[(STEP 1)]
			\item Find all $x_i^\ast$ s.t. $f^\prime (x_i^\ast)=0$ (critical points), $f^{\prime\prime}(x_i^\ast)=0$, boundary points, convergence toward undefined points, and tail ($i.e.$, convergence toward $\pm\infty$)
			\item Calculate $f(x_i^\ast)$
			\item Make table for graph sketch
		\end{enumerate}
	\end{block}
	\[
		\tag{Ex3.6} f(x) = \frac{16(x+1)}{(x-2)^2}
	\]
\end{frame}
%--- Next Frame ---%
% section graphing_rational_functions (end)

\section{Tails and Horizontal Asymptotes} % (fold)
\label{sec:tails_and_horizontal_asymptotes}
\begin{frame}[t]{Tail}
	\begin{block}
		{Tails of Polynomial}
		Only two cases: diverge to $\pm\infty$
	\end{block}
	\begin{block}
		{Tails of Rational Function}
		\[
			g(x) = \frac{\bar a_0 x^{\bar k} + \bar a_1 x^{\bar k-1} + \cdots + \bar a_{\bar k}}{\bar b_0 x^{\bar m} + \cdots +\bar b_{\bar m}}
		\]
		Tails of $g(x)$ is determined by $\frac{\bar a_0}{\bar b_0}\frac{x^{\bar k}}{x^{\bar m}}$
		\begin{itemize}
			\item $k>m$: Same as polynomials with degree $k-m$
			\item $k=m$: Converges to $\frac{a_0}{b_0}$ (Horizontal asymptote)
			\item $k<m$: converges to $0$ (Horizontal asymptote)
		\end{itemize}
	\end{block}
\end{frame}
%--- Next Frame ---%
% section tails_and_horizontal_asymptotes (end)

\section{Maxima and Minima} % (fold)
\label{sec:maxima_and_minima}
\begin{frame}[t]{Boundary Max and Interior Max}
	\begin{thm}
		[3.3: First Order Condition (FOC)]
		$x_0$ is an interior max or min of $f$ $\Rightarrow$ $x_0$ is a critical point of $f$. $i.e.,$ $f^\prime(x_0)=0$
		(Inverse is not always true)
	\end{thm}
	
	\begin{thm}
		[3.4: Second Order Condition (SOC)]
		\begin{enumerate}
			\item $f^\prime(x_0)=0 \land f^{\prime\prime}(x_0)<0 \Rightarrow x_0$ is local max of $f$
			\item $f^\prime(x_0)=0 \land f^{\prime\prime}(x_0)>0 \Rightarrow x_0$ is local min of $f$
			\item $f^\prime(x_0)=0 \land f^{\prime\prime}(x_0)=0 \Rightarrow x_0$ can be max, min, or neither
		\end{enumerate}
	\end{thm}
\end{frame}
%--- Next Frame ---%

\begin{frame}[t]{Global Maxima and Minima}
	\begin{itemize}
		\item Finding global max (or min) is not easy problem
		\item These cases guarantee the existence of global max (or min)
		\begin{itemize}
			\item (1) Domain of $f$ is an interval $\land$ (2) $f$ has only one critical point $\land$ (3) $f^{\prime\prime}>0$ (g.min) $\lor$ $f^{\prime\prime}<0$ (g.max) in domain of $f$
			\item Domain of $f$ is compact (closed and bounded)    ($\exists$ global max, global min)
		\end{itemize}
		\item Below case guarantees the nonexistence of global max (or min)
		\begin{itemize}
			\item Strictly increasing (or decreasing) functions with open domain
		\end{itemize}
	\end{itemize}
\end{frame}
%--- Next Frame ---%
% section maxima_and_minima (end)

\section{Applications to Economics} % (fold)
\label{sec:applications_to_economics}
\begin{frame}[t]{Producer's Problem in Perfect Competative Market}
	
	\begin{block}
		{Producer's Problem in perfect competitive market}
		\[
			\arg\max_x \Pi(x)
		\]
	\end{block}
	\[
		x=f(L) \tag{Production Function}
	\]
	\begin{block}
		{Exogenous (Given) variables}
		\begin{itemize}
			\item $\bar w$: unit price of labor
			\item $\bar p$: unit price of end product
		\end{itemize}
	\end{block}
\end{frame}
%--- Next Frame ---%

\begin{frame}[t]{Assumptions}
	\begin{block}
		{Assumptions}
		\begin{itemize}
			\item $f:D\rightarrow\mathbb{R}\subset \mathbf{C}^2$
			\item $f$ is increasing: $f^\prime(L) >0 \forall L\in D$
			\item $\exists \bar a\ge 0$ s.t. (1) $f^{\prime\prime}(L)>0 \forall L\in[0,\bar a)$ ($i.e.,$ convex on $[0,\bar a)$) and (2) $f^{\prime\prime}(L)<0 \forall L\in(\bar a,\infty)$ ($i.e.,$ concave on $(\bar a,\infty)$)
			\item Quantity of input (labor) $L$ is the only factor for production
		\end{itemize}
	\end{block}

\end{frame}
%--- Next Frame ---%


\begin{frame}[t]{Cost Functions}
	\begin{block}
		{Big Picture for problem solving}
		Production Function $\rightarrow$ Cost Function (in terms of $x$) $\rightarrow$ Profit Function $\Pi(x)$ $\rightarrow$ Finding $x^\ast$ maximizing $\Pi(x)$
	\end{block}
	
	\begin{defn}
		[Total Cost, Marginal Cost, and Average Cost]
		\begin{itemize}
			\item $TC(x)$: Total cost for producing $x$
			\item $MC(x):= TC^\prime(x)$
			\item $AC(x):= \frac{TC(x)}{x}$
		\end{itemize}
	\end{defn}
	
	\begin{thm}
		[3.7c]
		At interior minimum of AC ($i.e.,AC^\prime=0$ ), $AC=MC$
	\end{thm}
\end{frame}
%--- Next Frame ---%

\begin{frame}[t]{Revenue and Profit Functions}
	\begin{defn}
		[Total Revenue, Marginal Revenue]
		\[
			TR(x) := \bar p x
		\]
		\[
			MR(x) := TR^\prime (x)
		\]
	\end{defn}
	\begin{defn}
		[Profit Function]
		\[
			\Pi(x) := TR-TC
		\]
	\end{defn}
\end{frame}
%--- Next Frame ---%

\begin{frame}[t]{Producer's problem in Monopoly Case}
	In monopoly, $p$ is endogenous
	\begin{block}
		{Producer's Problem in Monopoly}
		\[
			\arg\max_{p,x} \Pi(x)
		\]
	\end{block}
	However, firm is facing demand directly in monopoly
	\[
		x=D(p) \tag{Demand Function}
	\]
\end{frame}
%--- Next Frame ---%

\begin{frame}[t]{Elasticity}
	\begin{defn}
		[A Elasticity of B]
		\[
			\epsilon_{B,A} := \frac{\frac{d B}{B}}{\frac{d A}{A}} = \frac{A}{B}\frac{dB}{dA}
		\]
	\end{defn}
	\begin{itemize}
		\item $\frac{\Delta x}{x}$: rate of change
		\item Elasticity: ratio of rate of change
		\begin{itemize}
			\item $|\epsilon|<1$: inelastic
			\item $|\epsilon|>1$: elastic
			\item $|\epsilon|=1$: unit elastic
		\end{itemize}
	\end{itemize}
\end{frame}
%--- Next Frame ---%

\begin{frame}[t]{Functions with Constant Demand}
	\begin{itemize}
		\item Elasticity of linear demand function is not constant (not realistic)
		\[
			x=D(p)=\bar a - \bar b p,\quad \bar a,\bar b >0
		\]
		\item Example of constant elasticity demand function (more realistic)
		\[
			x=D(p)=\bar k p^{-\bar r},\quad \bar k, \bar r >0
		\]
	\end{itemize}
\end{frame}
%--- Next Frame ---%

% section applications_to_economics (end)
\end{document}
