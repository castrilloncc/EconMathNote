\documentclass[a4paper,11pt]{article}
\usepackage[hyperref]{beamerarticle}

% \documentclass[final]{beamer}

\usepackage{kotex}
\usepackage{amsfonts,amsmath,xob-amssymb}

\usepackage{amsthm}
\newtheorem{defn}{Definition}
\newtheorem{thm}{Theorem}

\usepackage{cancel}
\usepackage{enumerate}

\mode<presentation>{
	\usetheme{Madrid}
	\usecolortheme{default}
	\usefonttheme{professionalfonts}
}

\def\b{\boldsymbol}

\mode<article>{
\usepackage{fullpage}
}
\usepackage{ulem}

\newcommand{\bb}{\mathbb}
\newcommand{\bd}{\mathbf}
\newcommand{\p}{\partial}

\newcommand{\mail}{\url{mailto:eyeofyou@korea.ac.kr}}

\author[조남운]{\mail}
\title{Quadratic Forms and Definite Matrices}
\subtitle{Ch.16}

\begin{document}

\maketitle

\mode<presentation>{
\begin{frame}[t]{Table of Contents}
	\tableofcontents
\end{frame}
%--- Next Frame ---%
}

\section{Quadratic Forms} % (fold)
\label{sec:quadratic_forms}
\begin{frame}[t]{Quadratic Forms}
	\begin{defn}
		[Quadratic Form]
		A \uline{quadratic form} on $\bb{R}^n$ is a real-valued function of the form\[
			Q(\bd{x})=\bd{x}^T A \bd{x},\quad \bd{x}\in\bb{R}^n,\quad A^T=A
		\]
	\end{defn}
	For more detailed description, see Ch13 (section 3).
\end{frame}
%--- Next Frame ---%
% section quadratic_forms (end)

\section{Definiteness of Quadratic Forms} % (fold)
\label{sec:definiteness_of_quadratic_forms}
\begin{frame}[t]{Definiteness}
	\begin{block}
		{Definiteness: Overview}
		When $Q=\bd{x}^TA\bd{x}$ and $A$ is a diagonal matrix\[
			A = \begin{pmatrix}
				a_{11} & 0 & 0&\cdots& 0\\
				0 & a_{22} & 0&\cdots& 0\\
				\vdots &\vdots &\vdots &\vdots &\vdots\\
				0 & 0 & 0 & \cdots & a_{nn}
			\end{pmatrix}
		\]
		\begin{itemize}
			\item Positive Definite (PD): $a_{ii}>0 \quad\forall i$
			\item Positive Semi Definite (PSD): $a_{ii}\ge 0 \quad\forall i$
			\item Negative Definite (ND): $a_{ii}<0 \quad\forall i$
			\item Negative Semi Definite (NSD): $a_{ii}\le 0 \quad\forall i$
			\item Indefinite (ID): $a_{ii}<0 $ for some $i$, and $a_{ii}>0 $ for some $i$
		\end{itemize}
	\end{block}
\end{frame}
%--- Next Frame ---%

\begin{frame}[t]{Definite Symmetric Matrices}
	\begin{defn}
		[PD,PSD,ND,NSD,ID]
		Let $A$ be an $n\times n$ symmetric matrix and $Q=\bd{x}^T A \bd{x}$, then $A$ is:
		\begin{enumerate}
			\item PD if $Q>0$ $\forall\bd{x}\neq\bd{0}$ in $\bb{R}^n$
			\item PSD if $Q\ge0$ $\forall\bd{x}\neq\bd{0}$ in $\bb{R}^n$
			\item ND if $Q<0$ $\forall\bd{x}\neq\bd{0}$ in $\bb{R}^n$
			\item NSD if $Q\le 0$ $\forall\bd{x}\neq\bd{0}$ in $\bb{R}^n$
			\item ID if $Q>0$ for some $\bd{x}\in \bb{R}^n$ and $Q<0$ for some $\bd{x}\in\bb{R}^n$
		\end{enumerate}
	\end{defn}
\end{frame}
%--- Next Frame ---%

\begin{frame}[t]{Principal Minors of a Matrix}
	\begin{defn}
		[Principal Submatrix (PS), Principal Minor (PM)]
		Let $A$ be an $n\times n$ symmetric matrix. $k$th order \uline{principal submatrix} of $A$ is $k\times k$ submatrix of $A$ obtained by deleting $n-k$ columns $C_1,\cdots,C_{n-k}$ and same $n-k$ rows $R_1,\cdots,R_{n-k}$.

		$k$th order \uline{principal minor} of $A$ is the determinant of $k$th order prinicipal submatrix.
	\end{defn}
	Note: the number of $k$th order principal submatrix can be $nCk$. We will denote $k$th order principal minor by $PM_k(A)$.
	\begin{defn}
		[Leading PS, Leading PM]
		$k$th order \uline{leading principal submatrix} of $A$ is an unique $k$th order submatrix obtained by deleting the last $n-k$ rows and columns from $A$. $k$th order \uline{leading principal minor} of A ($LPM_k(A)$  or $|A_k|$) is the determinant of $k$th order leading principal submatrix of $A$
	\end{defn}
\end{frame}
%--- Next Frame ---%

\begin{frame}[t]{Test for Definiteness}
	\begin{thm}
		[16.1,2] Let $A$ be an $n\times n$ symmetric matrix. Then,
		\begin{enumerate}
			\item $A$ is PD iff $LPM_k(A)>0$ $\forall k$
			\item $A$ is PSD iff $PM_k(A)\ge 0$ $\forall k$
			\item $A$ is ND iff ${\rm sign}(LPM_k(A))={\rm{sign}}((-1)^k)\quad \forall k$
			\item $A$ is NSD iff ${\rm sign}(PM_k(A))={\rm{sign}}((-1)^k)\quad \forall PM_k(A)\neq 0$
			\item Otherwise, $A$ is ID
		\end{enumerate}
	\end{thm}
	Note: We can find more elegant criteria using eigenvalues (Ch.23). To check all $PM$, $\sum_i^n nCi$ determinants should be calculated.
\end{frame}
%--- Next Frame ---%
% section definiteness_of_quadratic_forms (end)

\section{Linear Constraints and Bordered Matrices} % (fold)
\label{sec:linear_constraints_and_bordered_matrices}
\begin{frame}[t]{Bordered Matrix}
	\begin{block}
		{Finding global max/min of $Q(x_1,x_2)$ with one linear constraint}
		\[
			Q(x_1,x_2)=\bd{x}^T H \bd{x} = \begin{pmatrix}
				x_1 x_2
			\end{pmatrix}\begin{pmatrix}
				a & b\\
				b & c
			\end{pmatrix}\begin{pmatrix}
				x_1\\
				x_2
			\end{pmatrix} = ax_1^2 + 2bx_1x_2 + cx_2^2
		\] on \[
			\begin{pmatrix}
				A & B
			\end{pmatrix}\begin{pmatrix}
				x_1\\
				x_2
			\end{pmatrix}=G\bd{x}=0
		\]
		Substitute $x_1$ to $-Bx_2/A$ and we can get one variable function $\tilde Q$ in terms of $x_2$\[
			\tilde{Q}(x_2) = Q(-Bx_2/A,x_2)=\frac{aB^2-2bAB+cA^2}{A^2}x_2^2
		\] \[
			aB^2-2bAB+cA^2 = -\det \begin{pmatrix}
				0&A&B\\
				A&a&b\\
				B&b&c
			\end{pmatrix}=-\det \begin{pmatrix}
				0&G\\
				G^T&H
			\end{pmatrix}
		\]
	\end{block}
\end{frame}
%--- Next Frame ---%

\begin{frame}[t]{Definiteness of Bordered Matrix}
	\begin{thm}
		[16.3] $Q(\bd{x})$ is PD[ND] on the constraint set $G\bd{x}=0$ iff\[
			\det \begin{pmatrix}
							0&A&B\\
							A&a&b\\
							B&b&c
						\end{pmatrix}=\det \begin{pmatrix}
							0&G\\
							G^T&H
						\end{pmatrix}
		\] is negative[positive]
	\end{thm}
	Note: sign of determinant is dependent on both $n$ (size of $\bd{x}$) and $m$ (number of restriction)
\end{frame}
%--- Next Frame ---%

\begin{frame}[t]{General Bordered Matrix}
	Consider a general quadratic form \[
				Q(\bd{x})=\bd{x}^T A \bd{x} = \begin{pmatrix}
					x_1 &\cdots& x_n
				\end{pmatrix}\begin{pmatrix}
					a_{11} & \cdots & a_{1n}\\
					\vdots & \vdots & \vdots\\
					a_{n1} & \cdots & a_{nn}
				\end{pmatrix}\begin{pmatrix}
					x_1\\\vdots\\x_n
				\end{pmatrix}
			\] with linear constraint set\[
				B\bd{x}=\begin{pmatrix}
					b_{11}&\cdots&b_{1n}\\
					\vdots&\vdots&\vdots\\
					b_{m1}&\cdots&b_{mn}
				\end{pmatrix}\begin{pmatrix}
					x_1\\\vdots\\x_n
				\end{pmatrix}=\begin{pmatrix}
					0\\\vdots\\0
				\end{pmatrix}.
			\]
			We can make $(n+m)\times(n+m)$ symmetric matrix (general bordered matrix) \[
				H = \begin{pmatrix}
					\bd{0}&B\\
					B^T&A
				\end{pmatrix}
			\]
\end{frame}
%--- Next Frame ---%

\begin{frame}[t]{Definiteness of General Bordered Matrix}
	\begin{thm}
		[16.4] To determine the definiteness of general bordered matrix, Check the signs of the last $n-m$ LPMs of $H$, starting with $LPM_{n+m}(H)$ ($i.e.,$ the determinant of $H$ itself). This means you should check the sign of \[
			\underbrace{LPM_{n+m}(H), LPM_{n+m-1}(H), \cdots,LPM_{n+m-(n-m-1)}(H)}_{n-m\text{ LPMs}}
		\]
		\begin{enumerate}[(a)]
			\item If ${\rm sign}(\det H)={\rm sign}((-1)^m)$ and all $n-m$ LPMs have same sign, $Q$ is \uline{PD} on the constraint set $B\bd{x}=\bd{0}$ and $\bd{x}=\bd 0$ is \uline{strict global min} of $Q$ on the costraint set
			\item If ${\rm sign}(\det H)={\rm sign}((-1)^n)$ and following $n-m$ LPMs alternates in sign, $Q$ is \uline{ND} on the constraint set $B\bd{x}=\bd{0}$ and $\bd{x}=\bd 0$ is \uline{strict global max} of $Q$ on the costraint set
		\end{enumerate}
	\end{thm}
\end{frame}
%--- Next Frame ---%

\begin{frame}[t]{Definiteness of General Bordered Matrix}
	\begin{block}
		{Continued}
		\begin{enumerate}[(a)]
			\item [(c)] if (a),(b) is violated by nonzero LPMs, $Q$ is \uline{ID} on the constraint set $B\bd{x}=\bd{0}$ and $\bd{x}=\bd 0$ is neither a max nor a min of $Q$ on the constraint set
		\end{enumerate}
	\end{block}
	Note: Test for NSD, PSD is much more tedious and trivial in economics $\rightarrow$ SKIP
\end{frame}
%--- Next Frame ---%
% section linear_constraints_and_bordered_matrices (end)

\end{document}
