% \documentclass[a4paper,11pt]{article}
% \usepackage[hyperref]{beamerarticle}
\documentclass[final]{beamer}

\usepackage{kotex}
\usepackage{amsfonts,amsmath,xob-amssymb}

\usepackage{amsthm}
\newtheorem{defn}{Definition}
\newtheorem{thm}{Theorem}

\usepackage{cancel}
\usepackage{enumerate}

\mode<presentation>{
	\usetheme{Madrid}
	\usecolortheme{default}
	\usefonttheme{professionalfonts}
}

\def\b{\boldsymbol}

\mode<article>{
\usepackage{fullpage}
}
\usepackage{ulem}

\newcommand{\bb}{\mathbb}
\newcommand{\bd}{\mathbf}
\newcommand{\p}{\partial}

\newcommand{\mail}{\url{econMath.namun+2016sp@gmail.com}}

\author[조남운]{\mail}
\title{Quadratic Forms and Definite Matrices}
\subtitle{Ch.16}

\begin{document}
	
\maketitle

\mode<presentation>{
\begin{frame}[t]{Table of Contents}
	\tableofcontents
\end{frame}
%--- Next Frame ---%
}

\section{Quadratic Forms} % (fold)
\label{sec:quadratic_forms}
\begin{frame}[t]{Quadratic Forms}
	\begin{defn}
		[Quadratic Form]
		A \uline{quadratic form} on $\bb{R}^n$ is a real-valued function of the form\[
			Q(\bd{x})=\bd{x}^T A \bd{x},\quad \bd{x}\in\bb{R}^n,\quad A^T=A
		\]
	\end{defn}
	For more detailed description, see Ch13 (section 3).
\end{frame}
%--- Next Frame ---%
% section quadratic_forms (end)

\section{Definiteness of Quadratic Forms} % (fold)
\label{sec:definiteness_of_quadratic_forms}
\begin{frame}[t]{Definiteness}
	\begin{block}
		{Definiteness: Overview}
		When $Q=\bd{x}^TA\bd{x}$ and $A$ is a diagonal matrix\[
			A = \begin{pmatrix}
				a_{11} & 0 & 0&\cdots& 0\\
				0 & a_{22} & 0&\cdots& 0\\
				\vdots &\vdots &\vdots &\vdots &\vdots\\
				0 & 0 & 0 & \cdots & a_{nn} 
			\end{pmatrix}
		\]
		\begin{itemize}
			\item Positive Definite (PD): $a_{ii}>0 \quad\forall i$
			\item Positive Semi Definite (PSD): $a_{ii}\ge 0 \quad\forall i$
			\item Negative Definite (ND): $a_{ii}<0 \quad\forall i$
			\item Negative Semi Definite (NSD): $a_{ii}\le 0 \quad\forall i$
			\item Indefinite (ID): $a_{ii}<0 $ for some $i$, and $a_{ii}>0 $ for some $i$
		\end{itemize}
	\end{block}
\end{frame}
%--- Next Frame ---%

\begin{frame}[t]{Definite Symmetric Matrices}
	\begin{defn}
		[PD,PSD,ND,NSD,ID]
		Let $A$ be an $n\times n$ symmetric matrix and $Q=\bd{x}^T A \bd{x}$, then $A$ is:
		\begin{enumerate}
			\item PD if $Q>0$ $\forall\bd{x}\neq\bd{0}$ in $\bb{R}^n$
			\item PSD if $Q\ge0$ $\forall\bd{x}\neq\bd{0}$ in $\bb{R}^n$
			\item ND if $Q<0$ $\forall\bd{x}\neq\bd{0}$ in $\bb{R}^n$
			\item NSD if $Q\le 0$ $\forall\bd{x}\neq\bd{0}$ in $\bb{R}^n$
			\item ID if $Q>0$ for some $\bd{x}\in \bb{R}^n$ and $Q<0$ for some $\bd{x}\in\bb{R}^n$
		\end{enumerate}
	\end{defn}
\end{frame}
%--- Next Frame ---%

\begin{frame}[t]{Principle Minors of a Matrix}
	\begin{defn}
		[Principal Submatrix, Principal Minor]
		Let $A$ be an $n\times n$ symmetric matrix. $k$th order \uline{principal submatrix} of $A$ is $k\times k$ submatrix of $A$ obtained by deleting $n-k$ columns $C_1,\cdots,C_{n-k}$ and same $n-k$ rows $R_1,\cdots,R_{n-k}$. 
		
		$k$th order \uline{principal minor} of $A$ is the determinant of $k$th order prinicipal submatrix.
	\end{defn}
	Note: the number of $k$th order principal submatrix can be $nCk$. 
\end{frame}
%--- Next Frame ---%
% section definiteness_of_quadratic_forms (end)

\section{Linear Constraints and Bordered Matrices} % (fold)
\label{sec:linear_constraints_and_bordered_matrices}

% section linear_constraints_and_bordered_matrices (end)

\end{document}

