\documentclass[a4paper,11pt]{article}
\usepackage[hyperref]{beamerarticle}

% \documentclass[final]{beamer}

\usepackage[hangul]{kotex}
\usepackage{amsfonts,amsmath,xob-amssymb}

\usepackage{amsthm}
\newtheorem{defn}{Definition}
\newtheorem{thm}{Theorem}

\usepackage{cancel}
\usepackage{enumerate}

\mode<presentation>{
	\usetheme{Madrid}
	\usecolortheme{default}
	\usefonttheme{professionalfonts}
}

\def\b{\boldsymbol}

\mode<article>{
\usepackage{fullpage}
}
\usepackage{ulem}

\newcommand{\bb}{\mathbb}
\newcommand{\bd}{\mathbf}
\newcommand{\p}{\partial}

\newcommand{\mail}{\url{mailto:eyeofyou@korea.ac.kr}}

\author[조남운]{\mail}
\title{One-Variable Calculus: Chain Rule}
\subtitle{CH4}

\begin{document}

	\maketitle

\mode<presentation>{
\begin{frame}[t]{Table of Contents}
	\tableofcontents
\end{frame}
}
%--- Next Frame ---%

\section{Composite Functions and the Chain Rule} % (fold)
\label{sec:composite_functions_and_the_chain_rule}

\begin{frame}[t]{Derivative of Composite Functions}
	\begin{defn}
		[Composition of Functions]
		\[
			( h\circ g )(x) := h(g(x))
		\]
		$h$: outside function, $g$: inside function
	\end{defn}
	In general, $h\circ g \neq g \circ h$
	\begin{block}
		{Chain Rule}
		Suppose $\hat f_i = (f_i \circ f_{i+1} \circ \cdots \circ f_n)(x)=f_i(f_{i+1}(\cdots(f_n(x))))$. then,
		\[
			\frac{d( f_1 \circ f_2 \circ \cdots \circ f_n)(x)}{dx}=\frac{df_1(f_2(\cdots(f_n(x))))}{dx}= \frac{d\hat f_1}{d\hat f_2}\frac{d\hat f_2}{d\hat f_3}\cdots\frac{d\hat f_n}{dx}
		\]
	\end{block}
		Exercise: $[\cos((\sin(x^3+4x))^{500})]^\prime =?$
\end{frame}
%--- Next Frame ---%

% section composite_functions_and_the_chain_rule (end)


\section{Inverse Functions and Their Derivatives} % (fold)
\label{sec:inverse_functions_and_their_derivatives}

\begin{frame}[t]{Inverse of a Function}
	\begin{defn}
		[Inverse Function]
		Suppose $f:E_1 \rightarrow E_2$. Then $g:E_2\rightarrow E_1$ is an \uline{inverse} of $f$ if\[
			g(f(x))=x\quad \forall x\in E_1
		\]\[
						f(g(x))=x\quad \forall x\in E_2
		\]
	\end{defn}
	\begin{itemize}
		\item Notation: $g(x)=f^{-1}(x)$
		\item $(f^{-1}\circ f) (x) = (f\circ f^{-1}) (x)=x$
		\item Geometrical meaning: Graph of $f^{-1}$ is reflection of the graph of $f$ across 45 degree line
		\item $\exists f^{-1}$ ($i.e.$, $f$ is invertible) iff $f$ is strictly and monotonically increasing [decreasing]
	\end{itemize}
\end{frame}
%--- Next Frame ---%

\begin{frame}[t]{Derivative of the Inverse Function}
	\begin{thm}
		[4.3: Inverse Function Theorem] Suppose $f:I\rightarrow \mathbb{R} \text{ is } \mathbf{C}^1$ Function, $f^\prime \neq 0, x\in I$. Then,
		\begin{enumerate}
			\item $\exists f^{-1}$ on $I$
			\item $f^{-1}\in \mathbf{C}^1$ on interval $f(I)$
			\item $(f^{-1})^\prime = \frac{1}{f^\prime(f^{-1})}$. more intuitively, 
			\[
				\frac{df^{-1}}{dx} = \frac{1}{\frac{df(f^{-1})}{df^{-1}}}=\frac{1}{\frac{dx}{df^{-1}}}
			\]
		\end{enumerate}
	\end{thm}
	\[
		f(f^{-1})^\prime = x^\prime = 1 \quad\Rightarrow\quad \frac{df(f^{-1})}{dx}=1 \quad\Rightarrow\quad \frac{df(f^{-1})}{df^{-1}}\frac{df^{-1}}{dx}=1
	\]
\end{frame}
%--- Next Frame ---%

\begin{frame}[t]{The Derivative of $x^{m/n}$}
	\begin{thm}
		[4.4] \[
			\forall n\in\mathbb{N},\quad (x^{1/n})^\prime = \frac{1}{n} x^{(1/n)-1}
		\]
	\end{thm}
	\begin{thm}
		[4.5] \[
			\forall m,n\in\mathbb{N},\quad (x^{m/n})^\prime = \frac{m}{n} x^{(m/n)-1}
		\]
	\end{thm}
	In general, 
	\[
		(x^r)^\prime = rx^{(r-1)}\quad \forall r\in\mathbb{R}
	\]
\end{frame}
%--- Next Frame ---%

% section inverse_functions_and_their_derivatives (end)
\end{document}
