\documentclass[a4paper,11pt]{article}
\usepackage[hyperref]{beamerarticle}

%\documentclass[final]{beamer}

\usepackage{kotex}
\usepackage{amsfonts,amsmath,xob-amssymb}

\usepackage{amsthm}
\newtheorem{defn}{Definition}
\newtheorem{thm}{Theorem}

\usepackage{cancel}
\usepackage{enumerate}

\mode<presentation>{
	\usetheme{Madrid}
	\usecolortheme{default}
	\usefonttheme{professionalfonts}
}

\def\b{\boldsymbol}

\mode<article>{
\usepackage{fullpage}
}
\usepackage{ulem}

\newcommand{\mail}{\url{econMath.namun+2016su@gmail.com}}

\author[조남운]{\mail}
\title{Functions of Several Variables}
\subtitle{Ch.13}

\begin{document}
	
\maketitle

\mode<presentation>{
\begin{frame}[t]{Table of Contents}
	\tableofcontents
\end{frame}
%--- Next Frame ---%
}

\section{Functions Between Euclidean Spaces} % (fold)
\label{sec:functions_between_euclidean_spaces}
\begin{frame}[t]{Definitions}
	\begin{defn}
		[Function, Domain, Target, Image: General Definitions]
		\uline{Function} $f:A\rightarrow B$ is a rule that assigns each object of $A$ (\uline{domain}) to one object in $B$ (\uline{target space}). \uline{Image} of $f$ is $\{f(\mathbf{x})\vert \mathbf{x}\in A\}\subset B$
	\end{defn}
	\begin{block}
		{Examples: $f:\mathbb{R}^n\rightarrow\mathbb{R}$}
		\begin{align*}
			f(\mathbf{x})&=\bar{\mathbf{a}}\bullet\mathbf{x}\tag{Linear}\\
			f(\mathbf{x})&=\bar k \prod_i x_i^{\bar b_i} \tag{Cobb-Douglas}\\
			f(\mathbf{x})&=\bar k \left(\sum_i \bar c_i x_i ^{-\bar a}\right)^{-\bar b / \bar a} \tag{CES}
		\end{align*}
	\end{block}
\end{frame}
%--- Next Frame ---%

\begin{frame}[t]{$\mathbf{f}:\mathbb{R}^k\rightarrow\mathbb{R}^m$}
	\begin{block}
		{$\mathbf{f}:\mathbb{R}^k\rightarrow\mathbb{R}^m$}
		Let $f_i: \mathbb{R}^n\rightarrow\mathbb{R}$. Then $\mathbf{f}:\mathbb{R}^k\rightarrow\mathbb{R}^m$ can be represented by $f_i$ ($i=1,2,\cdots,m$)
		\[
			\mathbf{f(x)}:=(f_1(\mathbf{x}),\cdots,f_m(\mathbf{x}))
		\]
	\end{block}
	\begin{block}
		{Examples}
		\begin{itemize}
			\item Production function with $k$ input factors and $m$ products
			\item Utility mapping $\mathbf{u}: \mathbb{R}^{km}\rightarrow\mathbb{R}^m$
			\begin{itemize}
				\item $u_i:\mathbb{R}^k\rightarrow\mathbb{R}$: Individual utility function of customer $i$
				\item $k$: \# of goods
				\item $m$: \# of consumers
				\item $\mathbf{x_i}$: consumption of customer $i$
				\[
					\mathbf{u}(\mathbf{x_1},\cdots,\mathbf{x_m})=(u_1(\mathbf{x_1}),\cdots,u_m(\mathbf{x_m}))
				\]
			\end{itemize}
		\end{itemize}
	\end{block}
\end{frame}
%--- Next Frame ---%
\begin{frame}[t]{$\mathbf{f}:\mathbb{R}\rightarrow\mathbb{R}^m$}
	\begin{block}
		{$\mathbf{f}(t)$}
		\[
			\mathbf{f}(t):=\left(f_1(t),\cdots,f_m(t)\right)
		\]
	\end{block}
	Geometrically, $\mathbf{f}(t)$ is a parametric curve on $\mathbb{R}^m$ space (cf. parametric line)
\end{frame}
%--- Next Frame ---%
% section functions_between_euclidean_spaces (end)

\section{Geometric Representation of Functions} % (fold)
\label{sec:geometric_representation_of_functions}
\begin{frame}[t]{Level Curves}
	\begin{block}
		{Level Curves}
		Let $f:\mathbb{R}^2\rightarrow\mathbb{R}^1$. Then \uline{level curve}s of $f$ are curves on domain space with same $f(\mathbf{x})$. \textit{I.e.,}
		\[
			\{\mathbf{x}\vert f(\mathbf{x})=\bar c\}
		\]
		\begin{itemize}
			\item Isoquant: level curve of production function
			\item Indifference curve: level curve of utility function
			\item Generally, when $f:\mathbb{R}^k\rightarrow\mathbb{R}$ it is called \uline{level set} and this is $k$ dimensional nonlinear object
		\end{itemize}
	\end{block}
\end{frame}
%--- Next Frame ---%
% section geometric_representation_of_functions (end)

\section{Special Kinds of Functions} % (fold)
\label{sec:special_kinds_of_functions}
\begin{frame}[t]{Linear Functions on $\mathbb{R}^k$}
	\begin{defn}
		[Linear Function from $\mathbf{R}^k$ to $\mathbf{R}^m$]
		$\mathbf{f}$ is a \uline{linear function} when
		\begin{enumerate}
			\item $\mathbf{f}(\mathbf{x_1}+\mathbf{x_2})=\mathbf{f}(\mathbf{x_1})+\mathbf{f}(\mathbf{x_2})$
			\item $\mathbf{f}(r\mathbf{x})=r\mathbf{f(x)}$
		\end{enumerate}
	\end{defn}
	\begin{thm}
		[13.1,2] \begin{itemize}
			\item $f:\mathbb{R}^k\rightarrow\mathbb{R}$ is a linear function $\Rightarrow$ $f(\mathbf{x})=\bar{\mathbf{a}}\bullet\mathbf{x}$, $\mathbf{a}\in\mathbb{R}^k$
			\item $\mathbf{f}:\mathbb{R}^k\rightarrow\mathbb{R}^m$ is a linear function $\Rightarrow$ $\mathbf{f}(\mathbf{x})=\bar A\bullet\mathbf{x}$, $A:m\times k$ matrix
		\end{itemize}
		
	\end{thm}
\end{frame}
%--- Next Frame ---%

\begin{frame}[t]{Quadratic Forms}
	\begin{defn}
		[Quadratic Form on $\mathbb{R}^k$]
		$f:\mathbb{R}^k\rightarrow\mathbb{R}$ is of the \uline{quadratic form} if:
		\[
			f(\mathbf{x})=\sum_{i,j}^k \bar a_{ij} x_i x_j
		\]more elegantly, 
		\[
			f(\mathbf{x}) = \mathbf{x}^T \bar A \mathbf{x}
		\]
		In this case, $A$ can be always symmetric.
	\end{defn}
\end{frame}
%--- Next Frame ---%
\begin{frame}[t]{Monomial}
	\begin{defn}
		[Monomials on $\mathbb{R}^k$]
		$f:\mathbb{R}^k\rightarrow\mathbb{R}$ is a \uline{monomial} if:\[
			f(\mathbf{x})= \bar c \prod_{i}^k x_i^{\bar{a_i}}, \quad a_i\in\mathbb{N}\cup\{0\}
		\]
		The \uline{degree} of above monomial is $\sum_i^k a_i$
	\end{defn}
\end{frame}
%--- Next Frame ---%
% section special_kinds_of_functions (end)

\section{Continuous Functions} % (fold)
\label{sec:continuous_functions}
\begin{frame}[t]{Continuous}
	\begin{defn}
		[Continuous Function on $\mathbb{R}^k$]
		$\mathbf{f}=(f_1,\cdots,f_m)$ is \uline{continuous} at $\mathbf{x}$ iff all $f_i$ are continuous at $\mathbf{x}$
	\end{defn}
\end{frame}
%--- Next Frame ---%
% section continuous_functions (end)

% \section{Vocabulary of Functions} % (fold)
% \label{sec:vocabulary_of_functions}
%
% % section vocabulary_of_functions (end)
\end{document}

