\documentclass[a4paper]{article}

\usepackage{amsmath,xob-amssymb,amsfonts}
\usepackage[hangul]{kotex}
\title{Errata for Simon $et$ $al$. (1994)}
\author{Participants of Mathmatics for Economics Classes}

\begin{document}
\maketitle

\section{p.13} % (fold)
\label{sec:p_13}
\begin{itemize}
	\item $f_2(x) - x^7$ in caption for figure2.2 should be $f_2(x)=-x^7$ (2017sp 백지연)
\end{itemize}
% section p_13 (end)

\section{p.16} % (fold)
\label{sec:p_16}
$y=(x-1)/(x^3+3x+2)$ is just a typo. This equation should be
\[
	y=\frac{x-1}{x^2+3x+2}
\]
% section p_16 (end)

\section{p.49} % (fold)
\label{sec:p49}
\begin{itemize}
	\item 6th line: $5x+6$ should be $5x-6$ (2016f 송영석)
\end{itemize}
% section p49 (end)

\section{p.57} % (fold)
\label{sec:p_57}
In example 3.6, 
\[
	f(20) = 4500
\]
(2017sp 백지연)
% section p_57 (end)

\section{p.67} % (fold)
\label{sec:p67}
\begin{itemize}
	\item Figure 3.20: y intercept is $a$ (not $b$) (2016sp 이준현)
\end{itemize}
% section p67 (end)

\section{p.71} % (fold)
\label{sec:p_71}
\begin{itemize}
	\item In the Equation (2) of Example 4.2, $P(L) = \Pi(f(L))$ (2016f 송영석)
\end{itemize}
% section p_71 (end)

\section{p.177} % (fold)
\label{sec:p_177}
\begin{itemize}
	\item The last equation
	\[
		\mathbf{x}=(I-A)^{-1}
	\]
	should be
	\[
		\mathbf{x}=(I-A)^{-1}\mathbf{c}
	\]
	(2016f 배근태)
\end{itemize}
% section p_177 (end)

\section{p.192: Theorem 9.2} % (fold)
\label{sec:p_192_theorem_9_2}
\begin{itemize}
	\item Theorem 9.2 should be modified:
	
	\fbox{
	\begin{minipage}[t]{\textwidth}
		Let $A$ be an $n\times n$ matrix and let $R$ be its row echelon form by only using $ERO_1,ERO_2$. Then:
		\[
			\det A = \pm \det R
		\]
		If no row interchanges (\textit{i.e.}, $ERO_1$) and $ERO_3$ are used to compute $R$ from $A$, (or equivalently, if only $ERO_2$ are used,) then $\det A = \det R$
	\end{minipage}
	}
\end{itemize}

Proof sketch:
$\det(EM_1(R_i\leftrightarrow R_j))=-1, \det(EM_2(R_i\leftarrow R_i + kR_j))=1, \det(EM_3(R_i\leftarrow k R_i))=k$ and use Theorem 9.5(b)

% section p_192_theorem_9_2 (end)

\section{p.195} % (fold)
\label{sec:p_195}
\begin{itemize}
	\item In example 9.4, ``Example 7.1'' $\Rightarrow$ ``Exercise 7.2(b)'' (2017sp 백지연)
\end{itemize}
% section p_195 (end)

\section{p.275} % (fold)
\label{sec:p_275}
\begin{itemize}
	\item $q_1$ should be $q_2$ (2016sp 이준현)
	\[
		\mathbf{q}=(q_1,q_2)=\left(f_1(x_1,x_2,x_3),f_2(x_1,x_2,x_3)\right)\equiv F(x_1,x_2,x_3)
	\]
\end{itemize}
% section p_275 (end)

\section{p.321}
The first equation should be: \[
\begin{pmatrix}
	\frac{\partial F}{\partial x_1}(\mathbf{x}^\ast)\\
	\vdots\\
	\frac{\partial F}{\partial x_n}(\mathbf{x}^\ast)
\end{pmatrix}
\]

\section{p.327} % (fold)
\label{sec:p_327}
In theorem 14.4,
\[
	H = F \circ A : \mathbb{R}^s \rightarrow \mathbb{R}^m
\]
(2016su 박준현)
% section p_327 (end)

\section{p.337} % (fold)
\label{sec:p_337}
In Figure 15.2, two axis should be $x,y$, not $x_1, x_2$ (2016su 이가영)
% section p_337 (end)

\section{p.342} % (fold)
\label{sec:p_342}
In Theorem 15.2,

Then, there is a $C^1$ function $y=y(x_1,\cdots,x_k)$ defined on an open ball $B$ about ...

(2016su 박준후)
% section p_342 (end)

\section{p.349} % (fold)
\label{sec:p_349}
In Example 15.12, .. is perpendicular (or normal) to the plane\[
	Ax+By+Cz=D
\](2016su 이가영)
% section p_349 (end)

\section{p.400} % (fold)
\label{sec:p_400}
(In Theorem 17.3) Let $F:U\rightarrow \mathbb{R}^1$ be a $C^2$ function whose domain is an open set $U$ in $\mathbb{R}^n$.
(2016su 이은지)
% section p_400 (end)

\section{p.455} % (fold)
\label{sec:p_455}
In Equation 11, \[
	f(x^\ast(a);a)=f(a;a)= \cdots
\]
% section p_455 (end)

\section{p.458} % (fold)
\label{sec:p_458}

\begin{itemize}
	\item $D^2$ should be: (2016sp 이준현)
	\[
		D^2 f(\mathbf{x^\ast}) = \begin{pmatrix}
			\frac{\partial^2 f}{\partial x_1^2} & \cdots & \frac{\partial^2 f}{\partial x_nx_1}\\
			\vdots & \ddots & \vdots \\
			\frac{\partial^2 f}{\partial x_1x_n} & \cdots & \frac{\partial^2 f}{\partial x_n^2}
		\end{pmatrix}
	\]
\end{itemize}

% section p_458 (end)
\end{document}
